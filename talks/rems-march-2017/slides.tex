\documentclass[professionalfonts,smallfonts]{beamer}

\usepackage[british]{babel}
\usepackage{microtype}

\usetheme{metropolis}

\author{Dominic Mulligan \\ Joint work with Victor Gomes, Martin Kleppmann, and Alastair Beresford}
\title{Verifying Conflict Free Replicated Datatypes}
\date{Cambridge-Imperial REMS Meeting, 2017}

\begin{document}

\begin{frame}
\maketitle
\end{frame}

% What are CRDTs?
% Why are they useful?
% Why is verification necessary?
% What was the problem with previous proofs?
% Our proof: network model, abstract convergence, implementation of RGA
% Conclusion

\begin{frame}
\frametitle{Consistency models in distributed systems}
\emph{Strong consistency} model impossible or costly to achieve
\\[1.5ex]
Must seek weaker consistency guarantees
\\[1.5ex]
Popular alternative is \emph{eventual consistency}
\\[1.5ex]
Informally:
\begin{enumerate}
\item
If no changes are made to some component of shared global state
\item
Eventually all replicas will converge to some consensus on value of that component
\end{enumerate}
\end{frame}

\begin{frame}
\frametitle{Eventual consistency}
Eventual consistency is very weak consistency model!
\\[1.5ex]
Model:
\begin{itemize}
\item
makes no guarantees in case where updates never cease,
\item
and does not constrain return value of intermediate reads
\end{itemize}
\end{frame}

\begin{frame}
\frametitle{Trade-offs}
Range of consistency models exists because of a tradeoff:
\begin{center}
Performance vs. guarantees offered by model
\end{center}
\vspace{\baselineskip}
However, it is possible to:
\begin{itemize}
\item
Obtain stronger guarantees than those provided by eventual consistency
\item
Without inheriting the inherent costs of strong consistency
\end{itemize}
\vspace{\baselineskip}
One such consistency model is \emph{strong eventual consistency}
\end{frame}

\begin{frame}
\frametitle{Strong eventual consistency}
Strong eventual consistency (Shapiro, 2011) has:
\begin{itemize}
\item
Performance and fault-tolerance characteristics similar to eventual consistency
\item
Provides stronger guarantees
\end{itemize}
A system satisfying strong eventual consistency must satisfy:
\begin{itemize}
\item
Correct replicas that have received same updates have same state,
\item
Update delivered at a replica is eventually delivered to all replicas,
\item
All executions must terminate
\end{itemize}
First property (`convergence') most important
\end{frame}

\begin{frame}
\frametitle{Conflict Free Replicated Datatypes (CRDTs)}
Achieving consensus in SEC:
\begin{itemize}
\item
Typically requires conflict resolution policy
\item
Simple policies cause data loss, complex ones are error prone
\end{itemize}
Observation:
\begin{itemize}
\item
Some operations naturally commutative
\item
Operations can therefore be applied in any order
\item
Generalise this to other data structures and operations
\end{itemize}
CRDTs = replicated data types with commutative operations
\end{frame}

\begin{frame}
\frametitle{Three families of CRDT}
Three different types of CRDT are known:
\begin{itemize}
\item
Operation based CRDTs: construct operations so they are commutative
\item
State based CRDTs: broadcast entire state of replica when it changes
\item
Delta-CRDTs: broadcast changes to states, not entire states
\end{itemize}
We consider operation based CRDTs in this work.
\end{frame}

\begin{frame}
\frametitle{Why is verification necessary here?}
\emph{Operational Transformation} is a closely related technique
\\[1.5ex]
Algorithms have claimed to satisfy $TP_2$ but later shown incorrect
\\[1.5ex]
Algorithms have been presented without any proof
\\[1.5ex]
In classic formulation of OT, later shown impossible to achieve $TP_2$
\\[1.5ex]
Hand-written proofs of convergence tend to be gigantic
\\[1.5ex]
Several wrong machine checked proofs presented (!)
\\[1.5ex]
Moral: reasoning about replicated data structures is hard
\end{frame}

\begin{frame}
\frametitle{Diagnosing bugs in previous proofs}
Relied on high-level axioms stated in terms of data structure, and operations on it
\\[1.5ex]
But reasoning about data structure is hard, and unintuitive!
\\[1.5ex]
Wrong axioms are therefore hard to spot
\\[1.5ex]
Moral: need end-to-end verification to ensure correctness
\\[1.5ex]
Ensure required properties hold in all network executions
\end{frame}

\begin{frame}
\frametitle{Our contributions: a framework for verifying CRDTs}
We develop a stratified proof \emph{framework} for CRDTs
\\[1.5ex]
We have three different `components':
\begin{itemize}
\item
A network model,
\item
An abstract convergence theorem
\item
An implementation of a concrete CRDT
\end{itemize}
Piecing all three together gives a concrete convergence theorem for the CRDT
\\[1.5ex]
Two components are reusable
\end{frame}

\begin{frame}
\frametitle{Our contributions: modelling the network}
In one component, we:
\begin{itemize}
\item
Model a standard class of networks, axiomatically
\item
Our axioms are easy to defend
\item
Use standard broadcast/deliver event model for network
\item
Contents of messages are abstract
\end{itemize}
By working axiomatically:
\begin{itemize}
\item
Reason about all possible network behaviours
\item
Eliminate corner cases that may invalidate high-level proofs
\end{itemize}
\end{frame}

\begin{frame}
\frametitle{Our contributions: abstract convergence}
In another component, we:
\\[1.5ex]
Parameterise our work with a preorder on abstract `events' called \emph{happens before}
\\[1.5ex]
Assume a method of lifting `events' to `state transformers'
\\[1.5ex]
For two lists of events $xs$ and $ys$, if we assume:
\begin{itemize}
\item
Concurrent events in both $xs$ and $ys$ commute
\item
All events in $xs$ and $ys$ respect the happens before relation
\end{itemize}
Then: applying the state transformers in both $xs$ and $ys$ to the same initial state gives the same final state
\end{frame}

\begin{frame}
\frametitle{Our contributions}
In a final component, we:
\begin{itemize}
\item
Formalise the Replicated Growable Array (RGA) CRDT
\item
Show operations commute with themselves, and with each other, under assumptions
\item
Work in terms of concrete insert/delete messages with concrete payloads
\end{itemize}
\end{frame}

\begin{frame}
\frametitle{Composing components}
We compose:
\begin{itemize}
\item
Our network model has an associated concrete \emph{happens before} relation
\item
Given in terms of broadcast and deliver events
\item
This relation is a preorder
\end{itemize}
We can then replace the parameter in our abstract theorem with this relation
\end{frame}

\begin{frame}
\frametitle{Composing components}
As a corollary of abstract theorem, obtain a concrete convergence theorem for CRDT
\\[1.5ex]
Assuming two finite lists of deliver events, with insert/delete messages
\\[1.5ex]
Such that the messages delivered are the same (but not necessarily in same order)
\\[1.5ex]
Then applying these insert/delete operations to an initial state
\\[1.5ex]
Yields the same list
\end{frame}

\begin{frame}
\frametitle{Conclusions}
Distributed algorithms hard to verify
\\[1.5ex]
Many mistakes in previous proofs of distributed algorithms
\\[1.5ex]
Even those checked by machine!
\\[1.5ex]
Must use end-to-end verification for confidence
\\[1.5ex]
Developed a framework for end-to-end verification of CRDTs
\\[1.5ex]
Verified RGA CRDT implementation as an example
\end{frame}

\end{document}
