\section{Related work}\label{sect.relatedwork}

In a system where different replicas may concurrently perform updates without coordinating with each
other, strong eventual consistency (SEC, see Section~\ref{sect.eventual.consistency}) requires a
conflict resolution algorithm to reconcile concurrent updates. In some cases, a trivial algorithm is
used, for example:

\begin{description}
\item[User-defined conflict resolution:] Some systems store all conflicting versions of the data,
and either leave it for manual resolution by a user, or invoke a user-defined merge function.
However, manual resolution is an unacceptable burden for the user in many applications, and defining
merge functions in application code is error-prone; for example, \citet{DeCandia:2007ui} describe a
shopping cart anomaly at Amazon that arose due to poor conflict resolution.

\item[Last write wins (LWW):] Each version of the data structure is assigned a unique timestamp.
When there is a conflict, the system picks the version with the highest timestamp and discards other
versions. Although LWW achieves convergence, it does so at the cost of losing user input, which is
often unacceptable.
\end{description}

However, there are also algorithms that achieve convergence automatically, without discarding
updates. In Sections~\ref{sect.related.crdts} and~\ref{sect.related.ot} we summarise two main lines
of work, CRDTs and OT, which have the same fundamental goal of conflict resolution and convergence,
but which take different approaches towards achieving it.

\subsection{Conflict-free Replicated Data Types (CRDTs)}\label{sect.related.crdts}

Some operations, such as addition of numbers, are naturally commutative. Thus, if the replicated
data structure is a number whose value can only be incremented or decremented (a counter),
convergence can be achieved by applying the increment and decrement operations in any order at each
replica.

\emph{Conflict-free replicated data types} (CRDTs) generalise this idea to other data structures and
operations. For example, an ordered list (sequence) of values can be modified by inserting or
deleting elements at specified positions, and a map (dictionary) datatype can be modified by setting
the value associated with a key or deleting a key-value pair from the map. In a CRDT, those
modification operations are constructed to be commutative by attaching additional metadata to the
data structure.

To propagate changes between replicas, a CRDT either captures every update as an operation and
broadcasts it to other replicas (an \emph{operation-based} CRDT), or periodically broadcasts its
entire replica state (a \emph{state-based} CRDT). Operation-based CRDTs require operations to be
commutative; state-based CRDTs require a merge function over a join-semilattice, allowing two states
to be combined such that the result reflects changes made in both replicas. The two models have
different performance characteristics, but equivalent expressivity
\cite{Shapiro:2011wy,Shapiro:2011un}.

Many common abstract datatypes have been formulated as CRDTs, including
registers \cite{Shapiro:2011wy,Shapiro:2011un}, counters, maps \cite{Baquero:2016iv},
sets \cite{Bieniusa:2012wu,Bieniusa:2012gt}, XML \cite{Martin:2010ih},
and JSON trees \cite{Kleppmann:2016ve}. For ordered lists, several algorithms have been defined:
RGA \cite{Roh:2011dw}, Treedoc \cite{Preguica:2009fz}, WOOT \cite{Oster:2006wj},
Logoot \cite{Weiss:2010hx}, and LSEQ \cite{Nedelec:2013ky,Nedelec:2016eo}.
State-based CRDTs have been deployed commercially in the Riak database \cite{Brown:2014hs}.
Cloud types \cite{Burckhardt:2012jy} have similarities to CRDTs, using a relational data model.

In this work we formally verify three representative operation-based CRDTs: a counter
\cite{Shapiro:2011wy}, an OR-Set \cite{Bieniusa:2012gt}, and the RGA algorithm for ordered lists
\cite{Roh:2011dw}. While the counter and set are quite straightforward, RGA is subtle; we first
describe it informally in Section~\ref{sect.rga.background}, before showing how to formally prove
its convergence.


\subsection{Operational Transformation (OT)}\label{sect.related.ot}

Another family of algorithms for achieving convergence of replicas uses the \emph{operational
transformation} (OT) approach. They are designed for collaborative editing, that is, multiple users
concurrently modifying a document on their local device, and propagating updates asynchronously to
other users' devices. The replicated data structure is most commonly assumed to be a text document,
represented as an ordered list of characters that may be modified by inserting or deleting
characters at arbitrary positions in the string. OT algorithms for ordered lists include
dOPT \cite{Ellis:1989ue}, Jupiter \cite{Nichols:1995fd}, adOPTed \cite{Ressel:1996wx},
GOT \cite{Sun:1998un}, GOTO \cite{Sun:1998vf}, SOCT2 \cite{Suleiman:1997gl,Suleiman:1998eu},
SOCT3/4 \cite{Vidot:2000ch}, IMOR \cite{Imine:2003ks}, SDT \cite{Li:2004er,Li:2008hw}, and
TTF \cite{Oster:2006tr}.  The approach has also been generalised to other data structures such as
XML trees \cite{Ignat:2003jy,Davis:2002iv,Jungnickel:2015ua} and vector graphics documents
\cite{Sun:2002jb}.

Unlike CRDTs, in which update operations are commutative by definition, OT allows non-commutative
operations. Instead, OT relies on \emph{transforming} concurrent operations, allowing them to be
reordered on different nodes while ensuring a convergent outcome.

The required properties of this transformation function depend on assumptions about the network.
Many OT algorithms assume that operations are sequenced through a central server and delivered to
all clients in the same order. This design was originally pioneered by the Jupiter system
\cite{Nichols:1995fd} and is now used by all widely-deployed OT-based collaboration systems,
including Google Docs \cite{DayRichter:2010tt}, Microsoft Word Online, Etherpad
\cite{Etherpad:2011um}, Apache (formerly Google) Wave \cite{Wang:2015vo}, and Novell Vibe
\cite{Spiewak:2010vw}. With a central server, each client only needs to reorder its operations with
respect to the server's operation sequence, which simplifies the transformation.

On the other hand, as discussed in Section~\ref{sect.background.networks}, total order broadcast is
too restrictive for many peer-to-peer systems. Without total order broadcast, OT algorithms must
tolerate a higher degree of concurrency. Although a number of OT algorithms were purported to
guarantee convergence in peer-to-peer networks, many of them were later proved to be incorrect, as
discussed in the next section.

\subsection{Formal verification}\label{sect.related.verification}

Even though a data structure such as an ordered list may seem as though it ought to be simple, the
history of algorithms for achieving convergence in a distributed setting has been fraught with
difficulty. Informal reasoning has repeatedly produced approaches that fail to converge in certain
scenarios, and even several formal ``proofs'' later turned out to be false.

For OT, the properties that the transformation function must satisfy were first formalised by
\citet{Ressel:1996wx}. These properties are known as $\mathit{TP}_1$ and $\mathit{TP}_2$; systems
with a central server or total order broadcast need only satisfy $\mathit{TP}_1$, whereas
decentralised systems must satisfy both properties in order to ensure convergence. While
$\mathit{TP}_1$ has proved to be readily achievable in practice, and all the aforementioned
widely-deployed OT systems rely on it, the $\mathit{TP}_2$ property has been a significant source of
problems.

The original peer-reviewed publications of dOPT, adOPTed, IMOR, SOCT2, and SDT all claimed that
their transformation functions satisfied $\mathit{TP}_2$, but those claims were subsequently shown
to be false by giving counter-examples \cite{Imine:2003ks,Imine:2006kn,Oster:2005vi}. In the case of
dOPT and adOPTed, the $\mathit{TP}_2$ claim had originally been asserted without proof. In the case
of SOCT2 and SDT, there were hand-written ``proofs'' that later turned out to be incorrect. For IMOR
and SOCT2, there had even been machine-checked ``proofs'' \cite{Imine:2003ks}, but
\citet{Oster:2005vi} showed that they also were invalid because they made incorrect assumptions.

\citet{Randolph:2015gj} have even shown that in the classic formulation of OT it is impossible to
achieve $\mathit{TP}_2$. To our knowledge, TTF is at present the only $\mathit{TP}_2$-claiming OT
algorithm for which no counter-example is known, and it circumvents the impossibility result of
\citet{Randolph:2015gj} by using a different formulation of the transformation
\cite{Oster:2006tr,Levien:2016wz}.

Formal proofs of the $\mathit{TP}_1$ property have been more successful: \citet{Sinchuk:2016cf} and
\citet{Jungnickel:2015ua} verify transformation functions for trees, using Coq and Isabelle/HOL
respectively. For CRDTs, the only machine-checked verification of which we are aware is an Isabelle
formalisation of state-based sets, registers, and counters by \citet{Zeller:2014fl}; this work does
not consider any ordered list datatypes or any operation-based CRDTs.

The convergence of the RGA CRDT for ordered lists, which we study in this paper, has previously been
demonstrated in handwritten proofs \cite{Attiya:2016kh,Kleppmann:2016ve,Roh:2009ws}. Although we
have no reason to doubt the correctness of those proofs, the historic experience with
$\mathit{TP}_2$ makes us wary of claims whose assumptions and reasoning process have not been
checked rigorously. Other authors have also pointed out that handwritten proofs are laborious and
difficult to check by hand \cite{Li:2008hw,Li:2005jq}.

To our knowledge, our work is the first mechanised proof of operation-based CRDTs in general, and of
any ordered list CRDT in particular. As \citet{Oster:2005vi} have demonstrated, machine-checked
proofs are not immune to errors that are due to false assumptions. To avoid this trap, we prove not
only the commutativity of operations (which is subject to certain assumptions), but also that those
assumptions are guaranteed to hold in all executions of our network model. The network model in turn
is specified by a small set of axioms that are not specific to any particular CRDT, and whose
correctness can be robustly defended (see Section~\ref{sect.network}).

\citet{Burckhardt:2014ft} present a similar framework for reasoning about replicated datatypes, but
do not support mechanised proofs at present.
