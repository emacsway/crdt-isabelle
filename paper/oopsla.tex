\documentclass[acmlarge,review,anonymous]{acmart}\settopmatter{printfolios=true}

\bibliographystyle{ACM-Reference-Format}
\citestyle{acmauthoryear}
\usepackage[english]{babel}

\usepackage{isabelle,isabellesym}
\isabellestyle{it}

\setcopyright{none} % For review submission

\begin{document}
\title{Formal Verification of Peer-to-Peer Collaborative Editing}
%\author{Victor~B.~F.~Gomes, Martin Kleppmann, Dominic P.~Mulligan,\\Alastair R. Beresford}
%\date{Computer Laboratory, University of Cambridge}

\maketitle

\begin{abstract}
To be completed...
\end{abstract}

%%%%%%%%%%%%%%%%%%%%%%%%%%%%%%%%%%%%%%%%%%%%%%%%%%%%%%%%%%%%%%%%%%%%%%%%%%%%%%%%
% Introduction
%%%%%%%%%%%%%%%%%%%%%%%%%%%%%%%%%%%%%%%%%%%%%%%%%%%%%%%%%%%%%%%%%%%%%%%%%%%%%%%%

\section{Introduction}
\label{sect.introduction}

Collaborative editing applications such as Google Docs~\cite{DayRichter:2010tt}, Microsoft Word
Online, Etherpad~\cite{Etherpad:2011um}, and Novell Vibe~\cite{Spiewak:2010vw} are increasing in
popularity. A common feature of these tools is that they allow several users to concurrently modify
a document without having to send the document back and forth (e.g., by email), and without
requiring any exclusive locking or manual resolution of merge conflicts.

However, all currently deployed collaborative editing systems rely on a central server to determine
a sequential order of edit operations, a design originally pioneered by the Jupiter
system~\cite{Nichols:1995fd}. This architecture has the advantage of simplifying the collaborative
editing algorithm by restricting the concurrency in the system, but it has the downside of placing
significant trust in that single server: it is at risk of being compromised, censored, seized, or
otherwise subverted by adversaries. For sensitive scenarios, such as communication between
dissidents of a repressive regime, such centralization is problematic.

Decentralized peer-to-peer systems with end-to-end encryption can be more robust against such
interference. In this setting, especially when the participating nodes are mobile devices, totally
ordered broadcast of editing operations is prohibitively expensive~\cite{Attiya:2015dm}. Thus, there
has been significant interest in collaborative editing algorithms that work correctly in the face of
the increased concurrency encountered in peer-to-peer systems~\cite{Randolph:2015gj}.

There are two families of algorithms for collaborative editing: \emph{operational transformation}
(OT)~\cite{Ellis:1989ue,Ressel:1996wx,Oster:2006tr,Sun:1998vf,Sun:1998un,Suleiman:1998eu,Nichols:1995fd}
and \emph{conflict-free replicated datatypes}
(CRDTs)~\cite{Shapiro:2011wy,Roh:2011dw,Preguica:2009fz,Oster:2006wj,Weiss:2010hx,Nedelec:2013ky,Kleppmann:2016ve}.
Both allow a document to be modified concurrently on different replicas, with changes applied
immediately to the local copy, while asynchronously propagating changes to other replicas. The
goal of these algorithms is to ensure that for all concurrent executions, the replicas converge
toward the same state without any edits being lost (a property known as \emph{strong eventual
consistency}~\cite{Shapiro:2011un}).

However, these algorithms have a checkered history. OT algorithms have a reputation of being very
difficult to understand and to implement correctly~\cite{Spiewak:2010vw}. Despite the fact that OT
has been studied for almost three decades~\cite{Ellis:1989ue}, few algorithms work correctly in a
peer-to-peer setting, and several published algorithms were later shown to violate their supposed
convergence guarantees~\cite{Imine:2003ks,Imine:2006kn}. It has even been proved that in the classic
formulation of OT it is impossible to achieve the $\mathit{TP}_2$ property required for convergence
in a peer-to-peer setting, and that additional parameters must be added to transformation functions
to make convergence possible~\cite{Randolph:2015gj}.

CRDTs are a more recent development~\cite{Shapiro:2011un}. While OT is based on transforming
non-commutative operations so that they have the same effect when reordered, CRDTs define operations
in a way that makes them commutative by design, making them more amenable to peer-to-peer settings
in which each node may apply edits in a different order. CRDTs also have attractive performance
characteristics~\cite{Mehdi:2011ke}.

To date there has been fairly little formal verification of the correctness of CRDTs, and the
history of broken OT algorithms highlights the inadequacy of informal reasoning in this domain. In
this work we contribute to the formal basis of collaborative editing algorithms by using the
interactive proof assistant software Isabelle (TODO citation) to develop machine-checked proofs of
correctness for CRDTs.

In particular, we study the Replicated Growable Array (RGA) CRDT~\cite{Roh:2011dw}, which represents
a collaboratively editable document as a sequence of characters. There are previous pen-and-paper
correctness proofs of RGA in the literature~\cite{Attiya:2016kh,Kleppmann:2016ve,Roh:2009ws}, but to
our knowledge, ours is the first mechanized proof of the convergence of RGA. The algorithm is
subtle~-- Attiya et al.\ recently wrote, ``the reason why RGA actually works has been a bit of a
mystery''~\cite{Attiya:2016kh}~-- which makes formal verification particularly important.

Our proof is structured in four modules:
\begin{enumerate}
    \item A general convergence theorem that applies in any system where concurrent operations are
        commutative;
    \item A formal model of a network protocol providing reliable, causally-ordered broadcast;
    \item An implementation of the RGA algorithm, and a proof that well-formed, concurrent insertion
        and deletion operations commute;
    \item A proof that when the RGA algorithm is executed in our network model, all possible
        executions are well-formed, and thus converge.
\end{enumerate}

In doing so, we go much further than previous correctness proofs of collaborative editing
algorithms. Previous formalizations of OT using theorem
provers~\cite{Imine:2003ks,Imine:2006kn,Sinchuk:2016cf,Jungnickel:2015ua} focus on proving that the
transformation functions satisfy given properties (such as the transformation properties
$\mathit{TP}_1$ and $\mathit{TP}_2$~\cite{Oster:2006tr,Ressel:1996wx}), and do not explicitly model
the network. A previous formalization of CRDTs~\cite{Zeller:2014fl}, also using Isabelle, considers
other datatypes (sets, registers, counters) but not the ordered sequence datatype provided by RGA.

By including a model of the network in our proof, we rule out a larger set of potential errors in
the algorithm that may result from the interaction of operation properties with assumptions about
the network. Moreover, our network model and convergence theorems are independent of any particular
CRDT, so they can be reused for correctness proofs of any other replicated datatype that is based on
operation commutativity, encompassing a wide range of CRDTs~\cite{Baquero:2014ed}.

Besides presenting the first machine-checked proof of the RGA algorithm, our main contribution in
this paper is to establish a modular toolkit of proof techniques and building blocks for
machine-checked correctness proofs of operation-based CRDTs. Our proofs are broken down into modules
with well-defined properties, allowing modules to be reused for proofs of new datatypes in future.
By making formal verification easier, we hope to provide a strong foundation for the development of
the next generation of algorithms for collaborative editing.

%%%%%%%%%%%%%%%%%%%%%%%%%%%%%%%%%%%%%%%%%%%%%%%%%%%%%%%%%%%%%%%%%%%%%%%%%%%%%%%%
% Background
%%%%%%%%%%%%%%%%%%%%%%%%%%%%%%%%%%%%%%%%%%%%%%%%%%%%%%%%%%%%%%%%%%%%%%%%%%%%%%%%

\section{Background}
\label{sect.background}

%%%%%%%%%%%%%%%%%%%%%%%%%%%%%%%%%%%%%%%%%%%%%%%%%%%%%%%%%%%%%%%%%%%%%%%%%%%%%%%%
% Convergence
%%%%%%%%%%%%%%%%%%%%%%%%%%%%%%%%%%%%%%%%%%%%%%%%%%%%%%%%%%%%%%%%%%%%%%%%%%%%%%%%

\section{Our consistency model}
\label{sect.convergence}

Distributed systems---by their very nature---include replication and communication between replicas.
Messages passed back and forth between replicas across networks may be delayed, reordered, or lost entirely.
These effects may be caused by networking or other hardware faults, a conscious trade-off in the design of the networking protocol used, or they may simply be an inevitable result of geographic separation between replicas.
As a result, local state changes at any given pair of replicas may progress with differing views of the system's state.

A natural question to ask of a distributed system, therefore, is whether any two replicas will converge to the same state, and under what conditions that convergence will happen.
A \emph{consistency model} enumerates the criteria under which a distributed system converges, when, and how a replica's view of the system's state is related to the view of any other replica.
Several standard consistency models for distributed systems exist in the literature.
We discuss three here: \emph{strong consistency}, \emph{eventual consistency}, and a variant of the latter called \emph{strong eventual consistency}.

\paragraph{Strong consistency}

\paragraph{Eventual consistency} is a widely used consistency model in distributed systems, and is widely used in commercially deployed systems, particularly distributed data stores.
Barring no new updates 

\paragraph{Strong eventual consistency}

\emph{Strong eventual consistency} is a safety property that guarantees that any
two given nodes in a network that have received the same set of events converge to the same state.
We prove this convergence property in an abstract setting, that is, any two arbitrary list of distinct events, enriched by an order relation that satisfies a \emph{consistency property}, converge.

Let $\mathit{Events}$ be the set of events enriched with a preorder $\prec$. We
write $\isa{x} \prec \isa{y}$ and say that the event \isa{x} \emph{happens
before} the event \isa{y}.  Moreover we assume that there exists an
interpretation partial function $\langle\_\rangle$ that maps events to
\emph{operations}, which are \emph{state transformers} on a set of states
$\Sigma$.  This is implemented in Isabelle by the following \textbf{locale}
statement.

\begin{isabellebody}
  \isanewline
\isacommand{locale}\isamarkupfalse%
\ happens{\isacharunderscore}before\ {\isacharequal}\ preorder\ hb{\isacharunderscore}weak\ hb\isanewline
\ \ \isakeyword{for}\ hb{\isacharunderscore}weak\ {\isacharcolon}{\isacharcolon}\ {\isachardoublequoteopen}{\isacharprime}a\ {\isasymRightarrow}\ {\isacharprime}a\ {\isasymRightarrow}\ bool{\isachardoublequoteclose}\ \ {\isacharparenleft}\isakeyword{infix}\ {\isachardoublequoteopen}{\isasympreceq}{\isachardoublequoteclose}\ {\isadigit{5}}{\isadigit{0}}{\isacharparenright}\isanewline
\ \ \isakeyword{and}\ hb\ {\isacharcolon}{\isacharcolon}\ {\isachardoublequoteopen}{\isacharprime}a\ {\isasymRightarrow}\ {\isacharprime}a\ {\isasymRightarrow}\ bool{\isachardoublequoteclose}\ \ \ \ \ \ \ {\isacharparenleft}\isakeyword{infix}\ {\isachardoublequoteopen}{\isasymprec}{\isachardoublequoteclose}\ {\isadigit{5}}{\isadigit{0}}{\isacharparenright}\ {\isacharplus}\isanewline
\ \ \isakeyword{fixes}\ interp\ {\isacharcolon}{\isacharcolon}\ {\isachardoublequoteopen}{\isacharprime}a\ {\isasymRightarrow}\ {\isacharprime}b\ {\isasymrightharpoonup}\ {\isacharprime}b{\isachardoublequoteclose}\ {\isacharparenleft}{\isachardoublequoteopen}{\isasymlangle}{\isacharunderscore}{\isasymrangle}{\isachardoublequoteclose}\ {\isacharbrackleft}{\isadigit{0}}{\isacharbrackright}\ {\isadigit{1}}{\isadigit{0}}{\isadigit{0}}{\isadigit{0}}{\isacharparenright}\isanewline
\end{isabellebody}

We say that two events (or operations) $x$ and $y$ are \emph{concurrent}
whenever one does not happens before the other, that is, $\neg (\isa{x} \prec
\isa{y})$ and $\neg (\isa{y} \prec \isa{x})$.

\begin{isabellebody}
  \isanewline
\isacommand{definition}\isamarkupfalse%
\ concurrent\ {\isacharcolon}{\isacharcolon}\ {\isachardoublequoteopen}{\isacharprime}a\ {\isasymRightarrow}\ {\isacharprime}a\ {\isasymRightarrow}\ bool{\isachardoublequoteclose}\ {\isacharparenleft}\isakeyword{infix}\ {\isachardoublequoteopen}{\isasymparallel}{\isachardoublequoteclose}\ {\isadigit{5}}{\isadigit{0}}{\isacharparenright}\ \isakeyword{where}\isanewline
\ \ {\isachardoublequoteopen}x\ {\isasymparallel}\ y\ {\isasymequiv}\ {\isasymnot}\ {\isacharparenleft}x\ {\isasymprec}\ y{\isacharparenright}\ {\isasymand}\ {\isasymnot}\ {\isacharparenleft}y\ {\isasymprec}\ x{\isacharparenright}{\isachardoublequoteclose}\isanewline
\end{isabellebody}

A list of events \isa{xs} is \emph{hb-consistent}, or simply \emph{consistent},
whenever for any element \isa{y}, it is not the case that \isa{y} happens
before \isa{x} for all \isa{x} that appears before \isa{y} in the list. That
is, \isa{x} and \isa{y} are either concurrent or \isa{x} happens before
\isa{y}.  This is implemented in Isabelle by an inductive predicate.

\begin{isabellebody}
  \isanewline
\isacommand{inductive}\isamarkupfalse%
\ hb{\isacharunderscore}consistent\ {\isacharcolon}{\isacharcolon}\ {\isachardoublequoteopen}{\isacharprime}a\ list\ {\isasymRightarrow}\ bool{\isachardoublequoteclose}\ \isakeyword{where}\isanewline
\ \ {\isachardoublequoteopen}hb{\isacharunderscore}consistent\ {\isacharbrackleft}{\isacharbrackright}{\isachardoublequoteclose}\ {\isacharbar}\isanewline
\ \ {\isachardoublequoteopen}{\isasymlbrakk}\ hb{\isacharunderscore}consistent\ xs{\isacharsemicolon}\ {\isasymforall}x\ {\isasymin}\ set\ xs{\isachardot}\ {\isasymnot}\ y\ {\isasymprec}\ x\ {\isasymrbrakk}\ {\isasymLongrightarrow}\ hb{\isacharunderscore}consistent\ {\isacharparenleft}xs\ {\isacharat}\ {\isacharbrackleft}y{\isacharbrackright}{\isacharparenright}{\isachardoublequoteclose}\isanewline
\end{isabellebody}

Additionally, we say that concurrent operations commute in a list \isa{xs}
whenever for all concurrent events \isa{x} and \isa{y} in the list \isa{xs},
the Kleisli arrow composition of their operations commute $\langle \isa{x}
\rangle \rhd \langle \isa{y} \rangle = \langle \isa{y} \rangle \rhd \langle
\isa{x} \rangle$.

\begin{isabellebody}
  \isanewline
\isacommand{definition}\isamarkupfalse%
\ concurrent{\isacharunderscore}ops{\isacharunderscore}commute\ {\isacharcolon}{\isacharcolon}\ {\isachardoublequoteopen}{\isacharprime}a\ list\ {\isasymRightarrow}\ bool{\isachardoublequoteclose}\ \isakeyword{where}\isanewline
\ \ {\isachardoublequoteopen}concurrent{\isacharunderscore}ops{\isacharunderscore}commute\ xs\ {\isasymequiv}
\ {\isasymforall}x\ y{\isachardot}\ {\isacharbraceleft}x{\isacharcomma}\ y{\isacharbraceright}\ {\isasymsubseteq}\ set\ xs\ {\isasymlongrightarrow}\ x\ {\isasymparallel}\ y\ {\isasymlongrightarrow}\ {\isasymlangle}x{\isasymrangle}{\isasymrhd}{\isasymlangle}y{\isasymrangle}\ {\isacharequal}\ {\isasymlangle}y{\isasymrangle}{\isasymrhd}{\isasymlangle}x{\isasymrangle}{\isachardoublequoteclose}\isanewline

\isacommand{definition}\isamarkupfalse%
\ kleisli\ {\isacharcolon}{\isacharcolon}\ {\isachardoublequoteopen}{\isacharparenleft}{\isacharprime}b\ {\isasymRightarrow}\ {\isacharprime}b\ option{\isacharparenright}\ {\isasymRightarrow}\ {\isacharparenleft}{\isacharprime}b\ {\isasymRightarrow}\ {\isacharprime}b\ option{\isacharparenright}\ {\isasymRightarrow}\ {\isacharparenleft}{\isacharprime}b\ {\isasymRightarrow}\ {\isacharprime}b\ option{\isacharparenright}{\isachardoublequoteclose}\ {\isacharparenleft}\isakeyword{infixr}\ {\isachardoublequoteopen}{\isasymrhd}{\isachardoublequoteclose}\ {\isadigit{6}}{\isadigit{5}}{\isacharparenright}\ \isakeyword{where}\isanewline
\ \ {\isachardoublequoteopen}f\ {\isasymrhd}\ g\ {\isasymequiv}\ {\isasymlambda}x{\isachardot}\ f\ x\ {\isasymbind}\ {\isacharparenleft}{\isasymlambda}fx{\isachardot}\ g\ fx{\isacharparenright}{\isachardoublequoteclose}\isanewline
\end{isabellebody}

Let \isa{xs} be a list of events, \isa{apply-operations\ xs} is the state
transformer equivalent of applying in order each operation $\langle \isa{x}
\rangle$ for events \isa{x} in the list.  In other words, it is the fold of
the Kleisli arrow composition on the associated list of operations.

\begin{isabellebody}
  \isanewline
\isacommand{definition}\isamarkupfalse%
\ apply{\isacharunderscore}operations\ {\isacharcolon}{\isacharcolon}\ {\isachardoublequoteopen}{\isacharprime}a\ list\ {\isasymRightarrow}\ {\isacharprime}b\ {\isasymrightharpoonup}\ {\isacharprime}b{\isachardoublequoteclose}\ \isakeyword{where}\isanewline
\ \ {\isachardoublequoteopen}apply{\isacharunderscore}operations\ es\ s\ {\isasymequiv}\ {\isacharparenleft}foldl\ {\isacharparenleft}op\ {\isasymrhd}{\isacharparenright}\ Some\ {\isacharparenleft}map\ interp\ es{\isacharparenright}{\isacharparenright}\ s{\isachardoublequoteclose}\isanewline
\end{isabellebody}

\begin{isabellebody}
\isacommand{lemma}\isamarkupfalse%
\ concurrent{\isacharunderscore}ops{\isacharunderscore}commute{\isacharunderscore}concurrent{\isacharunderscore}set{\isacharcolon}\isanewline
\ \ \isakeyword{assumes}\ {\isachardoublequoteopen}concurrent{\isacharunderscore}ops{\isacharunderscore}commute\ {\isacharparenleft}prefix{\isacharat}suffix{\isacharat}{\isacharbrackleft}x{\isacharbrackright}{\isacharparenright}{\isachardoublequoteclose}\isanewline
\ \ \ \ \ \ \ \ \ \ {\isachardoublequoteopen}concurrent{\isacharunderscore}set\ x\ suffix{\isachardoublequoteclose}\isanewline
\ \ \ \ \ \ \ \ \ \ {\isachardoublequoteopen}distinct\ {\isacharparenleft}prefix\ {\isacharat}\ x\ {\isacharhash}\ suffix{\isacharparenright}{\isachardoublequoteclose}\isanewline
\ \ \isakeyword{shows}\ \ \ {\isachardoublequoteopen}apply{\isacharunderscore}operations\ {\isacharparenleft}prefix\ {\isacharat}\ suffix\ {\isacharat}\ {\isacharbrackleft}x{\isacharbrackright}{\isacharparenright}\ {\isacharequal}\ apply{\isacharunderscore}operations\ {\isacharparenleft}prefix\ {\isacharat}\ x\ {\isacharhash}\ suffix{\isacharparenright}{\isachardoublequoteclose}\isanewline
\end{isabellebody}

We can now formally state the abstract convergence theorem: two consistent list
of distinct events in which concurrent operations commute that have the same
set of elements yield the same state transformer.

\begin{isabellebody}
  \isanewline
\isacommand{theorem}\isamarkupfalse%
\ \ convergence{\isacharcolon}\isanewline
\ \ \isakeyword{assumes}\ {\isachardoublequoteopen}set\ xs\ {\isacharequal}\ set\ ys{\isachardoublequoteclose}\isanewline
\ \ \ \ \ \ \ \ \ \ {\isachardoublequoteopen}concurrent{\isacharunderscore}ops{\isacharunderscore}commute\ xs{\isachardoublequoteclose}\isanewline
\ \ \ \ \ \ \ \ \ \ {\isachardoublequoteopen}concurrent{\isacharunderscore}ops{\isacharunderscore}commute\ ys{\isachardoublequoteclose}\isanewline
\ \ \ \ \ \ \ \ \ \ {\isachardoublequoteopen}distinct\ xs{\isachardoublequoteclose}\isanewline
\ \ \ \ \ \ \ \ \ \ {\isachardoublequoteopen}distinct\ ys{\isachardoublequoteclose}\isanewline
\ \ \ \ \ \ \ \ \ \ {\isachardoublequoteopen}hb{\isacharunderscore}consistent\ xs{\isachardoublequoteclose}\isanewline
\ \ \ \ \ \ \ \ \ \ {\isachardoublequoteopen}hb{\isacharunderscore}consistent\ ys{\isachardoublequoteclose}\isanewline
\ \ \isakeyword{shows}\ \ \ {\isachardoublequoteopen}apply{\isacharunderscore}operations\ xs\ {\isacharequal}\ apply{\isacharunderscore}operations\ ys{\isachardoublequoteclose}\isanewline
\end{isabellebody}

The entire proof is shown in Figure~\ref{fig.convergence}.  The proof follows
by induction on \isa{xs}. The empty list case is immediate. Assuming the
property for \isa{xs}, we prove the case when a new event \isa{x} is added to
the end of \isa{xs}, that is, $\isa{xs}@[\isa{x}]$. Since both lists have the
same elements, one can split \isa{ys} such that \isa{ys = prefix@x@suffix} for
some suitable lists \isa{prefix} and \isa{suffix}. Moreover, all the events in
the set of \isa{suffix} is concurrent to \isa{x}. By the induction hypothesis,
\isa{apply-operations\ xs $=$ apply-operations\ (prefix $@$ suffix)}. Then
\begin{align*}
  \isa{apply-operations}\ (\isa{xs}@[\isa{x}])
  &= \langle\isa{x}\rangle\ (\isa{apply-operations}\ \isa{xs}) \\
  &= \langle\isa{x}\rangle\ (\isa{apply-operations}\ (\isa{prefix}@\isa{suffix}))\\
  &= \isa{apply-operations} (\isa{prefix}@\isa{suffix}@\isa{x}) \\
  &= \isa{apply-operations} (\isa{prefix}@\isa{x@\isa{suffix}}) \\
  &= \isa{apply-operations}\ ys.
\end{align*}


\begin{figure}
  \raggedright
  \begin{isabellebody}
\isanewline
\isacommand{theorem}\isamarkupfalse%
\ \ convergence{\isacharcolon}\isanewline
\ \ \isakeyword{assumes}\ {\isachardoublequoteopen}set\ xs\ {\isacharequal}\ set\ ys{\isachardoublequoteclose}\isanewline
\ \ \ \ \ \ \ \ \ \ {\isachardoublequoteopen}concurrent{\isacharunderscore}ops{\isacharunderscore}commute\ xs{\isachardoublequoteclose}\isanewline
\ \ \ \ \ \ \ \ \ \ {\isachardoublequoteopen}concurrent{\isacharunderscore}ops{\isacharunderscore}commute\ ys{\isachardoublequoteclose}\isanewline
\ \ \ \ \ \ \ \ \ \ {\isachardoublequoteopen}distinct\ xs{\isachardoublequoteclose}\isanewline
\ \ \ \ \ \ \ \ \ \ {\isachardoublequoteopen}distinct\ ys{\isachardoublequoteclose}\isanewline
\ \ \ \ \ \ \ \ \ \ {\isachardoublequoteopen}hb{\isacharunderscore}consistent\ xs{\isachardoublequoteclose}\isanewline
\ \ \ \ \ \ \ \ \ \ {\isachardoublequoteopen}hb{\isacharunderscore}consistent\ ys{\isachardoublequoteclose}\isanewline
\ \ \isakeyword{shows}\ \ \ {\isachardoublequoteopen}apply{\isacharunderscore}operations\ xs\ {\isacharequal}\ apply{\isacharunderscore}operations\ ys{\isachardoublequoteclose}\isanewline
%
\isacommand{using}\isamarkupfalse%
\ assms\ \isacommand{proof}\isamarkupfalse%
{\isacharparenleft}induction\ xs\ arbitrary{\isacharcolon}\ ys\ rule{\isacharcolon}\ rev{\isacharunderscore}induct{\isacharcomma}\ simp{\isacharparenright}\isanewline
\ \ \isacommand{case}\isamarkupfalse%
\ assms{\isacharcolon}\ {\isacharparenleft}snoc\ x\ xs{\isacharparenright}\isanewline
\ \ \isacommand{then}\isamarkupfalse%
\ \isacommand{obtain}\isamarkupfalse%
\ prefix\ suffix\ \isakeyword{where}\ ys{\isacharunderscore}split{\isacharcolon}\ {\isachardoublequoteopen}ys\ {\isacharequal}\ prefix\ {\isacharat}\ x\ {\isacharhash}\ suffix\ {\isasymand}\ concurrent{\isacharunderscore}set\ x\ suffix{\isachardoublequoteclose}\isanewline
\ \ \ \ \isacommand{using}\isamarkupfalse%
\ hb{\isacharunderscore}consistent{\isacharunderscore}prefix{\isacharunderscore}suffix{\isacharunderscore}exists\ \isacommand{by}\isamarkupfalse%
\ fastforce\isanewline
\ \ \isacommand{moreover}\isamarkupfalse%
\ \isacommand{hence}\isamarkupfalse%
\ {\isacharasterisk}{\isacharcolon}\ {\isachardoublequoteopen}distinct\ {\isacharparenleft}prefix\ {\isacharat}\ suffix{\isacharparenright}{\isachardoublequoteclose}\ {\isachardoublequoteopen}hb{\isacharunderscore}consistent\ xs{\isachardoublequoteclose}\isanewline
\ \ \ \ \isacommand{using}\isamarkupfalse%
\ assms\ \isacommand{by}\isamarkupfalse%
\ auto\isanewline
\ \ \isacommand{moreover}\isamarkupfalse%
\ \isacommand{{\isacharbraceleft}}\isamarkupfalse%
\isanewline
\ \ \ \ \isacommand{have}\isamarkupfalse%
\ {\isachardoublequoteopen}hb{\isacharunderscore}consistent\ prefix{\isachardoublequoteclose}\ {\isachardoublequoteopen}hb{\isacharunderscore}consistent\ suffix{\isachardoublequoteclose}\isanewline
\ \ \ \ \ \ \isacommand{using}\isamarkupfalse%
\ ys{\isacharunderscore}split\ assms\ hb{\isacharunderscore}consistent{\isacharunderscore}append{\isacharunderscore}D{\isadigit{2}}\ hb{\isacharunderscore}consistent{\isacharunderscore}append{\isacharunderscore}elim{\isacharunderscore}ConsD\ \isacommand{by}\isamarkupfalse%
\ blast{\isacharplus}\isanewline
\ \ \ \ \isacommand{hence}\isamarkupfalse%
\ {\isachardoublequoteopen}hb{\isacharunderscore}consistent\ {\isacharparenleft}prefix\ {\isacharat}\ suffix{\isacharparenright}{\isachardoublequoteclose}\isanewline
\ \ \ \ \ \ \isacommand{by}\isamarkupfalse%
\ {\isacharparenleft}metis\ assms{\isacharparenleft}{\isadigit{8}}{\isacharparenright}\ hb{\isacharunderscore}consistent{\isacharunderscore}append\ hb{\isacharunderscore}consistent{\isacharunderscore}append{\isacharunderscore}porder\ list{\isachardot}set{\isacharunderscore}intros{\isacharparenleft}{\isadigit{2}}{\isacharparenright}\ ys{\isacharunderscore}split{\isacharparenright}\isanewline
\ \ \isacommand{{\isacharbraceright}}\isamarkupfalse%
\isanewline
\ \ \isacommand{moreover}\isamarkupfalse%
\ \isacommand{have}\isamarkupfalse%
\ {\isacharasterisk}{\isacharasterisk}{\isacharcolon}\ {\isachardoublequoteopen}concurrent{\isacharunderscore}ops{\isacharunderscore}commute\ {\isacharparenleft}prefix\ {\isacharat}\ suffix\ {\isacharat}\ {\isacharbrackleft}x{\isacharbrackright}{\isacharparenright}{\isachardoublequoteclose}\isanewline
\ \ \ \ \isacommand{using}\isamarkupfalse%
\ assms\ ys{\isacharunderscore}split\ \isacommand{by}\isamarkupfalse%
\ {\isacharparenleft}clarsimp\ simp{\isacharcolon}\ concurrent{\isacharunderscore}ops{\isacharunderscore}commute{\isacharunderscore}def{\isacharparenright}\isanewline
\ \ \isacommand{moreover}\isamarkupfalse%
\ \isacommand{hence}\isamarkupfalse%
\ {\isachardoublequoteopen}concurrent{\isacharunderscore}ops{\isacharunderscore}commute\ {\isacharparenleft}prefix\ {\isacharat}\ suffix{\isacharparenright}{\isachardoublequoteclose}\isanewline
\ \ \ \ \isacommand{by}\isamarkupfalse%
\ {\isacharparenleft}force\ simp\ del{\isacharcolon}\ append{\isacharunderscore}assoc\ simp\ add{\isacharcolon}\ append{\isacharunderscore}assoc{\isacharbrackleft}symmetric{\isacharbrackright}{\isacharparenright}\isanewline
\ \ \isacommand{ultimately}\isamarkupfalse%
\ \isacommand{have}\isamarkupfalse%
\ {\isachardoublequoteopen}apply{\isacharunderscore}operations\ xs\ {\isacharequal}\ apply{\isacharunderscore}operations\ {\isacharparenleft}prefix{\isacharat}suffix{\isacharparenright}{\isachardoublequoteclose}\isanewline
\ \ \ \ \isacommand{using}\isamarkupfalse%
\ assms\ \isacommand{by}\isamarkupfalse%
\ simp\ {\isacharparenleft}metis\ Diff{\isacharunderscore}insert{\isacharunderscore}absorb\ Un{\isacharunderscore}iff\ {\isacharasterisk}\ concurrent{\isacharunderscore}ops{\isacharunderscore}commute{\isacharunderscore}appendD\ set{\isacharunderscore}append{\isacharparenright}\isanewline
\ \ \isacommand{moreover}\isamarkupfalse%
\ \isacommand{have}\isamarkupfalse%
\ {\isachardoublequoteopen}apply{\isacharunderscore}operations\ {\isacharparenleft}prefix{\isacharat}suffix\ {\isacharat}\ {\isacharbrackleft}x{\isacharbrackright}{\isacharparenright}\ {\isacharequal}\ apply{\isacharunderscore}operations\ {\isacharparenleft}prefix{\isacharat}x\ {\isacharhash}\ suffix{\isacharparenright}{\isachardoublequoteclose}\isanewline
\ \ \ \ \isacommand{using}\isamarkupfalse%
\ ys{\isacharunderscore}split\ assms\ {\isacharasterisk}{\isacharasterisk}\ concurrent{\isacharunderscore}ops{\isacharunderscore}commute{\isacharunderscore}concurrent{\isacharunderscore}set\ \isacommand{by}\isamarkupfalse%
\ force\isanewline
\ \ \isacommand{ultimately}\isamarkupfalse%
\ \isacommand{show}\isamarkupfalse%
\ {\isacharquery}case\isanewline
\ \ \ \ \isacommand{using}\isamarkupfalse%
\ ys{\isacharunderscore}split\ \isacommand{by}\isamarkupfalse%
\ {\isacharparenleft}force\ simp{\isacharcolon}\ append{\isacharunderscore}assoc{\isacharbrackleft}symmetric{\isacharbrackright}\ simp\ del{\isacharcolon}\ append{\isacharunderscore}assoc{\isacharparenright}\isanewline
\isacommand{qed}\isamarkupfalse%
  \end{isabellebody}
  \caption{Proof of convergence theorem in Isar.}
  \label{fig.convergence}
\end{figure}


%%%%%%%%%%%%%%%%%%%%%%%%%%%%%%%%%%%%%%%%%%%%%%%%%%%%%%%%%%%%%%%%%%%%%%%%%%%%%%%%
% Network
%%%%%%%%%%%%%%%%%%%%%%%%%%%%%%%%%%%%%%%%%%%%%%%%%%%%%%%%%%%%%%%%%%%%%%%%%%%%%%%%

\section{Network}
\label{sect.network}

\begin{isabellebody}
\isanewline
\isacommand{locale}\isamarkupfalse%
\ node{\isacharunderscore}histories\ {\isacharequal}\ \isanewline
\ \ \isakeyword{fixes}\ history\ {\isacharcolon}{\isacharcolon}\ {\isachardoublequoteopen}nat\ {\isasymRightarrow}\ {\isacharprime}a\ list{\isachardoublequoteclose}\isanewline
\ \ \isakeyword{assumes}\ histories{\isacharunderscore}distinct{\isacharcolon}\ {\isachardoublequoteopen}distinct\ {\isacharparenleft}history\ i{\isacharparenright}{\isachardoublequoteclose}\isanewline
\end{isabellebody}

\begin{isabellebody}
\isanewline
\isacommand{definition}\isamarkupfalse%
\ {\isacharparenleft}\isakeyword{in}\ node{\isacharunderscore}histories{\isacharparenright}\ history{\isacharunderscore}order\ {\isacharcolon}{\isacharcolon}\ {\isachardoublequoteopen}{\isacharprime}a\ {\isasymRightarrow}\ nat\ {\isasymRightarrow}\ {\isacharprime}a\ {\isasymRightarrow}\ bool{\isachardoublequoteclose}\ {\isacharparenleft}{\isachardoublequoteopen}{\isacharunderscore}{\isacharslash}\ {\isasymsqsubset}\isactrlsup {\isacharunderscore}{\isacharslash}\ {\isacharunderscore}{\isachardoublequoteclose}\ {\isacharbrackleft}{\isadigit{5}}{\isadigit{0}}{\isacharcomma}{\isadigit{1}}{\isadigit{0}}{\isadigit{0}}{\isadigit{0}}{\isacharcomma}{\isadigit{5}}{\isadigit{0}}{\isacharbrackright}{\isadigit{5}}{\isadigit{0}}{\isacharparenright}\ \isakeyword{where}\isanewline
\ \ {\isachardoublequoteopen}x\ {\isasymsqsubset}\isactrlsup i\ z\ {\isasymequiv}\ {\isasymexists}xs\ ys\ zs{\isachardot}\ xs{\isacharat}x{\isacharhash}ys{\isacharat}z{\isacharhash}zs\ {\isacharequal}\ history\ i{\isachardoublequoteclose}\isanewline
\end{isabellebody}

\begin{isabellebody}
\isanewline
\isacommand{definition}\isamarkupfalse%
\ {\isacharparenleft}\isakeyword{in}\ node{\isacharunderscore}histories{\isacharparenright}\ prefix{\isacharunderscore}of{\isacharunderscore}node{\isacharunderscore}history\ {\isacharcolon}{\isacharcolon}\ {\isachardoublequoteopen}{\isacharprime}a\ list\ {\isasymRightarrow}\ nat\ {\isasymRightarrow}\ bool{\isachardoublequoteclose}\ {\isacharparenleft}\isakeyword{infix}\ {\isachardoublequoteopen}prefix\ of{\isachardoublequoteclose}\ {\isadigit{5}}{\isadigit{0}}{\isacharparenright}\ \isakeyword{where}\isanewline
\ \ {\isachardoublequoteopen}xs\ prefix\ of\ i\ {\isasymequiv}\ {\isasymexists}ys{\isachardot}\ xs{\isacharat}ys\ {\isacharequal}\ history\ i{\isachardoublequoteclose}\isanewline
\isanewline
\end{isabellebody}

\begin{isabellebody}
\isanewline
\isacommand{datatype}\isamarkupfalse%
\ {\isacharprime}a\ event\isanewline
\ \ {\isacharequal}\ Broadcast\ {\isacharprime}a\isanewline
\ \ {\isacharbar}\ Deliver\ {\isacharprime}a\isanewline
\isanewline
\end{isabellebody}


\begin{isabellebody}
\isanewline
\isacommand{locale}\isamarkupfalse%
\ network\ {\isacharequal}\ node{\isacharunderscore}histories\ history\ \isakeyword{for}\ history\ {\isacharcolon}{\isacharcolon}\ {\isachardoublequoteopen}nat\ {\isasymRightarrow}\ {\isacharprime}a\ event\ list{\isachardoublequoteclose}\ {\isacharplus}\isanewline
\ \ \isakeyword{assumes}\ broadcast{\isacharunderscore}before{\isacharunderscore}delivery{\isacharcolon}\ {\isachardoublequoteopen}Deliver\ m\ {\isasymin}\ set\ {\isacharparenleft}history\ i{\isacharparenright}\ {\isasymLongrightarrow}\ {\isasymexists}j{\isachardot}\ Broadcast\ m\ {\isasymsqsubset}\isactrlsup j\ Deliver\ m{\isachardoublequoteclose}\isanewline
\ \ \isakeyword{and}\ deliver{\isacharunderscore}locally{\isacharcolon}\ {\isachardoublequoteopen}Broadcast\ m\ {\isasymin}\ set\ {\isacharparenleft}history\ i{\isacharparenright}\ {\isasymLongrightarrow}\ Broadcast\ m\ {\isasymsqsubset}\isactrlsup i\ Deliver\ m{\isachardoublequoteclose}\isanewline
\ \ \isakeyword{and}\ broadcasts{\isacharunderscore}unique{\isacharcolon}\ {\isachardoublequoteopen}i\ {\isasymnoteq}\ j\ {\isasymLongrightarrow}\ Broadcast\ m\ {\isasymin}\ set\ {\isacharparenleft}history\ i{\isacharparenright}\ {\isasymLongrightarrow}\ Broadcast\ m\ {\isasymnotin}\ set\ {\isacharparenleft}history\ j{\isacharparenright}{\isachardoublequoteclose}\isanewline
\end{isabellebody}

\begin{isabellebody}
\isanewline
\isacommand{inductive}\isamarkupfalse%
\ {\isacharparenleft}\isakeyword{in}\ network{\isacharparenright}\ hb\ {\isacharcolon}{\isacharcolon}\ {\isachardoublequoteopen}{\isacharprime}a\ {\isasymRightarrow}\ {\isacharprime}a\ {\isasymRightarrow}\ bool{\isachardoublequoteclose}\ \isakeyword{where}\isanewline
\ \ {\isachardoublequoteopen}{\isasymlbrakk}\ Broadcast\ m{\isadigit{1}}\ {\isasymsqsubset}\isactrlsup i\ Broadcast\ m{\isadigit{2}}\ {\isasymrbrakk}\ {\isasymLongrightarrow}\ hb\ m{\isadigit{1}}\ m{\isadigit{2}}{\isachardoublequoteclose}\ {\isacharbar}\isanewline
\ \ {\isachardoublequoteopen}{\isasymlbrakk}\ Deliver\ m{\isadigit{1}}\ {\isasymsqsubset}\isactrlsup i\ Broadcast\ m{\isadigit{2}}\ {\isasymrbrakk}\ {\isasymLongrightarrow}\ hb\ m{\isadigit{1}}\ m{\isadigit{2}}{\isachardoublequoteclose}\ {\isacharbar}\isanewline
\ \ {\isachardoublequoteopen}{\isasymlbrakk}\ hb\ m{\isadigit{1}}\ m{\isadigit{2}}{\isacharsemicolon}\ hb\ m{\isadigit{2}}\ m{\isadigit{3}}\ {\isasymrbrakk}\ {\isasymLongrightarrow}\ hb\ m{\isadigit{1}}\ m{\isadigit{3}}{\isachardoublequoteclose}\isanewline
\ \ \isanewline
\end{isabellebody}

\begin{isabellebody}
\isanewline
\isacommand{locale}\isamarkupfalse%
\ causal{\isacharunderscore}network\ {\isacharequal}\ network\ {\isacharplus}\isanewline
\ \ \isakeyword{assumes}\ causal{\isacharunderscore}delivery{\isacharcolon}\ {\isachardoublequoteopen}Deliver\ m{\isadigit{2}}\ {\isasymin}\ set\ {\isacharparenleft}history\ j{\isacharparenright}\ {\isasymLongrightarrow}\ hb\ m{\isadigit{1}}\ m{\isadigit{2}}\ {\isasymLongrightarrow}\ Deliver\ m{\isadigit{1}}\ {\isasymsqsubset}\isactrlsup j\ Deliver\ m{\isadigit{2}}{\isachardoublequoteclose}\isanewline
\isanewline
\end{isabellebody}

\begin{isabellebody}
\isanewline
\isacommand{definition}\isamarkupfalse%
\ {\isacharparenleft}\isakeyword{in}\ network{\isacharparenright}\ node{\isacharunderscore}deliver{\isacharunderscore}messages\ {\isacharcolon}{\isacharcolon}\ {\isachardoublequoteopen}{\isacharprime}a\ event\ list\ {\isasymRightarrow}\ {\isacharprime}a\ list{\isachardoublequoteclose}\ \isakeyword{where}\isanewline
\ \ {\isachardoublequoteopen}node{\isacharunderscore}deliver{\isacharunderscore}messages\ cs\ {\isasymequiv}\ List{\isachardot}map{\isacharunderscore}filter\ {\isacharparenleft}{\isasymlambda}e{\isachardot}\ case\ e\ of\ Deliver\ m\ {\isasymRightarrow}\ Some\ m\ {\isacharbar}\ {\isacharunderscore}\ {\isasymRightarrow}\ None{\isacharparenright}\ cs{\isachardoublequoteclose}\isanewline
\end{isabellebody}

\begin{isabellebody}
\isanewline
\isacommand{lemma}\isamarkupfalse%
\ {\isacharparenleft}\isakeyword{in}\ causal{\isacharunderscore}network{\isacharparenright}\isanewline
\ \ \isakeyword{shows}\ {\isachardoublequoteopen}hb{\isachardot}hb{\isacharunderscore}consistent\ {\isacharparenleft}node{\isacharunderscore}deliver{\isacharunderscore}messages\ {\isacharparenleft}history\ i{\isacharparenright}{\isacharparenright}{\isachardoublequoteclose}\isanewline
\end{isabellebody}

%%%%%%%%%%%%%%%%%%%%%%%%%%%%%%%%%%%%%%%%%%%%%%%%%%%%%%%%%%%%%%%%%%%%%%%%%%%%%%%%
% RGA
%%%%%%%%%%%%%%%%%%%%%%%%%%%%%%%%%%%%%%%%%%%%%%%%%%%%%%%%%%%%%%%%%%%%%%%%%%%%%%%%

\section{Replicated Growable Array}
\label{sect.rga}

\subsection{Operations}
\label{sect.rga.operations}

\begin{isabellebody}
\isanewline
\isacommand{type{\isacharunderscore}synonym}\isamarkupfalse%
\ {\isacharparenleft}{\isacharprime}id{\isacharcomma}\ {\isacharprime}v{\isacharparenright}\ elt\ {\isacharequal}\ {\isachardoublequoteopen}{\isacharprime}id\ {\isasymtimes}\ {\isacharprime}v\ {\isasymtimes}\ bool{\isachardoublequoteclose}%
\end{isabellebody}


\begin{isabellebody}
\isanewline
\isacommand{fun}\isamarkupfalse%
\ insert{\isacharunderscore}body\ {\isacharcolon}{\isacharcolon}\ {\isachardoublequoteopen}{\isacharparenleft}{\isacharprime}id{\isacharcolon}{\isacharcolon}{\isacharbraceleft}linorder{\isacharbraceright}{\isacharcomma}\ {\isacharprime}v{\isacharparenright}\ elt\ list\ {\isasymRightarrow}\ {\isacharparenleft}{\isacharprime}id{\isacharcomma}\ {\isacharprime}v{\isacharparenright}\ elt\ {\isasymRightarrow}\ {\isacharparenleft}{\isacharprime}id{\isacharcomma}\ {\isacharprime}v{\isacharparenright}\ elt\ list{\isachardoublequoteclose}\ \isakeyword{where}\isanewline
\ \ {\isachardoublequoteopen}insert{\isacharunderscore}body\ {\isacharbrackleft}{\isacharbrackright}\ \ \ \ \ e\ {\isacharequal}\ {\isacharbrackleft}e{\isacharbrackright}{\isachardoublequoteclose}\ {\isacharbar}\isanewline
\ \ {\isachardoublequoteopen}insert{\isacharunderscore}body\ {\isacharparenleft}x{\isacharhash}xs{\isacharparenright}\ e\ {\isacharequal}\isanewline
\ \ \ \ \ {\isacharparenleft}if\ fst\ x\ {\isacharless}\ fst\ e\ then\isanewline
\ \ \ \ \ \ \ \ e{\isacharhash}x{\isacharhash}xs\isanewline
\ \ \ \ \ \ else\ x{\isacharhash}insert{\isacharunderscore}body\ xs\ e{\isacharparenright}{\isachardoublequoteclose}\isanewline
\isanewline
\isacommand{fun}\isamarkupfalse%
\ insert\ {\isacharcolon}{\isacharcolon}\ {\isachardoublequoteopen}{\isacharparenleft}{\isacharprime}id{\isacharcolon}{\isacharcolon}{\isacharbraceleft}linorder{\isacharbraceright}{\isacharcomma}\ {\isacharprime}v{\isacharparenright}\ elt\ list\ {\isasymRightarrow}\ {\isacharparenleft}{\isacharprime}id{\isacharcomma}\ {\isacharprime}v{\isacharparenright}\ elt\ {\isasymRightarrow}\ {\isacharprime}id\ option\ {\isasymrightharpoonup}\ {\isacharparenleft}{\isacharprime}id{\isacharcomma}\ {\isacharprime}v{\isacharparenright}\ elt\ list{\isachardoublequoteclose}\ \isakeyword{where}\isanewline
\ \ {\isachardoublequoteopen}insert\ xs\ \ \ \ \ e\ None\ \ \ \ \ {\isacharequal}\ Some\ {\isacharparenleft}insert{\isacharunderscore}body\ xs\ e{\isacharparenright}{\isachardoublequoteclose}\ {\isacharbar}\isanewline
\ \ {\isachardoublequoteopen}insert\ {\isacharbrackleft}{\isacharbrackright}\ \ \ \ \ e\ {\isacharparenleft}Some\ i{\isacharparenright}\ {\isacharequal}\ None{\isachardoublequoteclose}\ {\isacharbar}\isanewline
\ \ {\isachardoublequoteopen}insert\ {\isacharparenleft}x{\isacharhash}xs{\isacharparenright}\ e\ {\isacharparenleft}Some\ i{\isacharparenright}\ {\isacharequal}\isanewline
\ \ \ \ \ {\isacharparenleft}if\ fst\ x\ {\isacharequal}\ i\ then\isanewline
\ \ \ \ \ \ \ \ Some\ {\isacharparenleft}x{\isacharhash}insert{\isacharunderscore}body\ xs\ e{\isacharparenright}\isanewline
\ \ \ \ \ \ else\isanewline
\ \ \ \ \ \ \ \ do\ {\isacharbraceleft}\ t\ {\isasymleftarrow}\ insert\ xs\ e\ {\isacharparenleft}Some\ i{\isacharparenright}\isanewline
\ \ \ \ \ \ \ \ \ \ \ {\isacharsemicolon}\ Some\ {\isacharparenleft}x{\isacharhash}t{\isacharparenright}\isanewline
\ \ \ \ \ \ \ \ \ \ \ {\isacharbraceright}{\isacharparenright}{\isachardoublequoteclose}\isanewline
\end{isabellebody}



\begin{isabellebody}
\isanewline
\isacommand{fun}\isamarkupfalse%
\ delete\ {\isacharcolon}{\isacharcolon}\ {\isachardoublequoteopen}{\isacharparenleft}{\isacharprime}id{\isacharcolon}{\isacharcolon}{\isacharbraceleft}linorder{\isacharbraceright}{\isacharcomma}\ {\isacharprime}v{\isacharparenright}\ elt\ list\ {\isasymRightarrow}\ {\isacharprime}id\ {\isasymrightharpoonup}\ {\isacharparenleft}{\isacharprime}id{\isacharcomma}\ {\isacharprime}v{\isacharparenright}\ elt\ list{\isachardoublequoteclose}\ \isakeyword{where}\isanewline
\ \ {\isachardoublequoteopen}delete\ {\isacharbrackleft}{\isacharbrackright}\ \ \ \ \ \ \ \ \ \ \ \ \ \ \ \ \ i\ {\isacharequal}\ None{\isachardoublequoteclose}\ {\isacharbar}\isanewline
\ \ {\isachardoublequoteopen}delete\ {\isacharparenleft}{\isacharparenleft}i{\isacharprime}{\isacharcomma}\ v{\isacharcomma}\ flag{\isacharparenright}{\isacharhash}xs{\isacharparenright}\ i\ {\isacharequal}\ \isanewline
\ \ \ \ \ {\isacharparenleft}if\ i{\isacharprime}\ {\isacharequal}\ i\ then\isanewline
\ \ \ \ \ \ \ \ Some\ {\isacharparenleft}{\isacharparenleft}i{\isacharprime}{\isacharcomma}\ v{\isacharcomma}\ True{\isacharparenright}{\isacharhash}xs{\isacharparenright}\isanewline
\ \ \ \ \ \ else\isanewline
\ \ \ \ \ \ \ \ do\ {\isacharbraceleft}\ t\ {\isasymleftarrow}\ delete\ xs\ i\isanewline
\ \ \ \ \ \ \ \ \ \ \ {\isacharsemicolon}\ Some\ {\isacharparenleft}{\isacharparenleft}i{\isacharprime}{\isacharcomma}v{\isacharcomma}flag{\isacharparenright}{\isacharhash}t{\isacharparenright}\isanewline
\ \ \ \ \ \ \ \ \ \ \ {\isacharbraceright}{\isacharparenright}{\isachardoublequoteclose}%
\end{isabellebody}


\begin{isabellebody}
\isanewline
\isacommand{lemma}\isamarkupfalse%
\ insert{\isacharunderscore}no{\isacharunderscore}failure{\isacharcolon}\isanewline
\ \ \isakeyword{assumes}\ {\isachardoublequoteopen}i\ {\isacharequal}\ None\ {\isasymor}\ {\isacharparenleft}{\isasymexists}i{\isacharprime}{\isachardot}\ i\ {\isacharequal}\ Some\ i{\isacharprime}\ {\isasymand}\ i{\isacharprime}\ {\isasymin}\ fst\ {\isacharbackquote}\ set\ xs{\isacharparenright}{\isachardoublequoteclose}\isanewline
\ \ \isakeyword{shows}\ \ \ {\isachardoublequoteopen}{\isasymexists}xs{\isacharprime}{\isachardot}\ insert\ xs\ e\ i\ {\isacharequal}\ Some\ xs{\isacharprime}{\isachardoublequoteclose}\isanewline
\end{isabellebody}


\begin{isabellebody}
\isanewline
\isacommand{lemma}\isamarkupfalse%
\ delete{\isacharunderscore}no{\isacharunderscore}failure{\isacharcolon}\isanewline
\ \ \isakeyword{assumes}\ {\isachardoublequoteopen}i\ {\isasymin}\ fst\ {\isacharbackquote}\ set\ xs{\isachardoublequoteclose}\isanewline
\ \ \isakeyword{shows}\ \ \ {\isachardoublequoteopen}{\isasymexists}xs{\isacharprime}{\isachardot}\ delete\ xs\ i\ {\isacharequal}\ Some\ xs{\isacharprime}{\isachardoublequoteclose}\isanewline
\end{isabellebody}


\begin{isabellebody}
\isanewline
\isacommand{lemma}\isamarkupfalse%
\ insert{\isacharunderscore}body{\isacharunderscore}preserve{\isacharunderscore}indices\ {\isacharbrackleft}simp{\isacharbrackright}{\isacharcolon}\isanewline
\ \ \isakeyword{shows}\ \ {\isachardoublequoteopen}fst\ {\isacharbackquote}\ set\ {\isacharparenleft}insert{\isacharunderscore}body\ xs\ e{\isacharparenright}\ {\isacharequal}\ fst\ {\isacharbackquote}\ set\ xs\ {\isasymunion}\ {\isacharbraceleft}fst\ e{\isacharbraceright}{\isachardoublequoteclose}\isanewline
\end{isabellebody}


\begin{isabellebody}
\isanewline
\isacommand{lemma}\isamarkupfalse%
\ delete{\isacharunderscore}preserve{\isacharunderscore}indices{\isacharcolon}\isanewline
\ \ \isakeyword{assumes}\ {\isachardoublequoteopen}delete\ xs\ i\ {\isacharequal}\ Some\ ys{\isachardoublequoteclose}\isanewline
\ \ \isakeyword{shows}\ \ \ {\isachardoublequoteopen}fst\ {\isacharbackquote}\ set\ xs\ {\isacharequal}\ fst\ {\isacharbackquote}\ set\ ys{\isachardoublequoteclose}\isanewline
\end{isabellebody}

\begin{isabellebody}
\isanewline
\isacommand{lemma}\isamarkupfalse%
\ insert{\isacharunderscore}commutes{\isacharcolon}\isanewline
\ \ \isakeyword{assumes}\ {\isachardoublequoteopen}fst\ e{\isadigit{1}}\ {\isasymnoteq}\ fst\ e{\isadigit{2}}{\isachardoublequoteclose}\isanewline
\ \ \ \ \ \ \ \ \ \ {\isachardoublequoteopen}i{\isadigit{1}}\ {\isacharequal}\ None\ {\isasymor}\ i{\isadigit{1}}\ {\isasymnoteq}\ Some\ {\isacharparenleft}fst\ e{\isadigit{2}}{\isacharparenright}{\isachardoublequoteclose}\isanewline
\ \ \ \ \ \ \ \ \ \ {\isachardoublequoteopen}i{\isadigit{2}}\ {\isacharequal}\ None\ {\isasymor}\ i{\isadigit{2}}\ {\isasymnoteq}\ Some\ {\isacharparenleft}fst\ e{\isadigit{1}}{\isacharparenright}{\isachardoublequoteclose}\isanewline
\ \ \isakeyword{shows}\ \ \ {\isachardoublequoteopen}do\ {\isacharbraceleft}\ ys\ {\isasymleftarrow}\ insert\ xs\ e{\isadigit{1}}\ i{\isadigit{1}}{\isacharsemicolon}\ insert\ ys\ e{\isadigit{2}}\ i{\isadigit{2}}\ {\isacharbraceright}\ {\isacharequal}\isanewline
\ \ \ \ \ \ \ \ \ \ \ \ \ do\ {\isacharbraceleft}\ ys\ {\isasymleftarrow}\ insert\ xs\ e{\isadigit{2}}\ i{\isadigit{2}}{\isacharsemicolon}\ insert\ ys\ e{\isadigit{1}}\ i{\isadigit{1}}\ {\isacharbraceright}{\isachardoublequoteclose}\isanewline
\end{isabellebody}


\begin{isabellebody}
\isanewline
\isacommand{lemma}\isamarkupfalse%
\ delete{\isacharunderscore}commutes{\isacharcolon}\isanewline
\ \ \isakeyword{shows}\ {\isachardoublequoteopen}do\ {\isacharbraceleft}\ ys\ {\isasymleftarrow}\ delete\ xs\ i{\isadigit{1}}{\isacharsemicolon}\ delete\ ys\ i{\isadigit{2}}\ {\isacharbraceright}\ {\isacharequal}\ do\ {\isacharbraceleft}\ ys\ {\isasymleftarrow}\ delete\ xs\ i{\isadigit{2}}{\isacharsemicolon}\ delete\ ys\ i{\isadigit{1}}\ {\isacharbraceright}{\isachardoublequoteclose}\isanewline
\end{isabellebody}


\begin{isabellebody}
\isanewline
\isacommand{lemma}\isamarkupfalse%
\ insert{\isacharunderscore}delete{\isacharunderscore}commute{\isacharcolon}\isanewline
\ \ \isakeyword{assumes}\ {\isachardoublequoteopen}i{\isadigit{1}}\ {\isacharequal}\ None\ {\isasymor}\ i{\isadigit{1}}\ {\isasymnoteq}\ Some\ {\isacharparenleft}fst\ e{\isacharparenright}{\isachardoublequoteclose}\isanewline
\ \ \ \ \ \ \ \ \ \ {\isachardoublequoteopen}i{\isadigit{2}}\ {\isasymnoteq}\ fst\ e{\isachardoublequoteclose}\isanewline
\ \ \isakeyword{shows}\ \ \ {\isachardoublequoteopen}do\ {\isacharbraceleft}\ ys\ {\isasymleftarrow}\ insert\ xs\ e\ i{\isadigit{1}}{\isacharsemicolon}\ delete\ ys\ i{\isadigit{2}}\ {\isacharbraceright}\ {\isacharequal}\ do\ {\isacharbraceleft}\ ys\ {\isasymleftarrow}\ delete\ xs\ i{\isadigit{2}}{\isacharsemicolon}\ insert\ ys\ e\ i{\isadigit{1}}\ {\isacharbraceright}{\isachardoublequoteclose}\isanewline
\end{isabellebody}


%%%%%%%%%%%%%%%%%%%%%%%%%%%%%%%%%%%%%%%%%%%%%%%%%%%%%%%%%%%%%%%%%%%%%%%%%%%%%%%%
% RGA_Network
%%%%%%%%%%%%%%%%%%%%%%%%%%%%%%%%%%%%%%%%%%%%%%%%%%%%%%%%%%%%%%%%%%%%%%%%%%%%%%%%

\subsection{RGA-Network}
\label{sect.rga.network}

\begin{isabellebody}
\isanewline
\isacommand{datatype}\isamarkupfalse%
\ {\isacharparenleft}{\isacharprime}id{\isacharcomma}\ {\isacharprime}v{\isacharparenright}\ operation\ {\isacharequal}\isanewline
\ \ Insert\ {\isachardoublequoteopen}{\isacharparenleft}{\isacharprime}id{\isacharcomma}\ {\isacharprime}v{\isacharparenright}\ elt{\isachardoublequoteclose}\ {\isachardoublequoteopen}{\isacharprime}id\ option{\isachardoublequoteclose}\ {\isacharbar}\isanewline
\ \ Delete\ {\isachardoublequoteopen}{\isacharprime}id{\isachardoublequoteclose}\isanewline
\end{isabellebody}

\begin{isabellebody}
\isanewline
\isacommand{fun}\isamarkupfalse%
\ interpret{\isacharunderscore}opers\ {\isacharcolon}{\isacharcolon}\ {\isachardoublequoteopen}{\isacharparenleft}{\isacharprime}id{\isacharcolon}{\isacharcolon}linorder{\isacharcomma}\ {\isacharprime}v{\isacharparenright}\ operation\ {\isasymRightarrow}\ {\isacharparenleft}{\isacharprime}id{\isacharcomma}\ {\isacharprime}v{\isacharparenright}\ elt\ list\ {\isasymrightharpoonup}\ {\isacharparenleft}{\isacharprime}id{\isacharcomma}\ {\isacharprime}v{\isacharparenright}\ elt\ list{\isachardoublequoteclose}\ {\isacharparenleft}{\isachardoublequoteopen}{\isasymlangle}{\isacharunderscore}{\isasymrangle}{\isachardoublequoteclose}\ {\isacharbrackleft}{\isadigit{0}}{\isacharbrackright}\ {\isadigit{1}}{\isadigit{0}}{\isadigit{0}}{\isadigit{0}}{\isacharparenright}\ \isakeyword{where}\isanewline
\ \ {\isachardoublequoteopen}interpret{\isacharunderscore}opers\ {\isacharparenleft}Insert\ e\ n{\isacharparenright}\ xs\ \ {\isacharequal}\ insert\ xs\ e\ n{\isachardoublequoteclose}\ {\isacharbar}\isanewline
\ \ {\isachardoublequoteopen}interpret{\isacharunderscore}opers\ {\isacharparenleft}Delete\ n{\isacharparenright}\ \ \ xs\ \ {\isacharequal}\ delete\ xs\ n{\isachardoublequoteclose}\isanewline
\end{isabellebody}

\begin{isabellebody}
\isanewline
\isacommand{locale}\isamarkupfalse%
\ rga\ {\isacharequal}\ network{\isacharunderscore}with{\isacharunderscore}ops\ {\isacharunderscore}\ interpret{\isacharunderscore}opers\ {\isacharplus}\isanewline
\ \ \isakeyword{assumes}\ insert{\isacharunderscore}flag{\isacharcolon}\ {\isachardoublequoteopen}Broadcast\ {\isacharparenleft}Insert\ e\ n{\isacharparenright}\ {\isasymin}\ set\ {\isacharparenleft}history\ i{\isacharparenright}\ {\isasymLongrightarrow}\ snd\ {\isacharparenleft}snd\ e{\isacharparenright}\ {\isacharequal}\ False{\isachardoublequoteclose}\isanewline
\ \ \isakeyword{assumes}\ allowed{\isacharunderscore}insert{\isacharcolon}\ {\isachardoublequoteopen}Broadcast\ {\isacharparenleft}Insert\ e\ n{\isacharparenright}\ {\isasymin}\ set\ {\isacharparenleft}history\ i{\isacharparenright}\ {\isasymLongrightarrow}\ n\ {\isacharequal}\ None\ {\isasymor}\ \isanewline
\ \ \ \ \ \ \ \ \ \ \ \ \ \ \ \ \ \ \ \ \ \ \ \ \ \ \ \ {\isacharparenleft}{\isasymexists}n{\isacharprime}\ n{\isacharprime}{\isacharprime}\ v\ b{\isachardot}\ n\ {\isacharequal}\ Some\ n{\isacharprime}\ {\isasymand}\ Deliver\ {\isacharparenleft}Insert\ {\isacharparenleft}n{\isacharprime}{\isacharcomma}\ v{\isacharcomma}\ b{\isacharparenright}\ n{\isacharprime}{\isacharprime}{\isacharparenright}\ {\isasymsqsubset}\isactrlsup i\ Broadcast\ {\isacharparenleft}Insert\ e\ n{\isacharparenright}{\isacharparenright}{\isachardoublequoteclose}\isanewline
\ \ \isakeyword{assumes}\ insert{\isacharunderscore}id{\isacharunderscore}unique{\isacharcolon}\ {\isachardoublequoteopen}id{\isadigit{1}}\ {\isacharequal}\ id{\isadigit{2}}\ {\isasymLongrightarrow}\ Broadcast\ {\isacharparenleft}Insert\ {\isacharparenleft}id{\isadigit{1}}{\isacharcomma}\ v{\isadigit{1}}{\isacharcomma}\ b{\isadigit{1}}{\isacharparenright}\ n{\isadigit{1}}{\isacharparenright}\ {\isasymin}\ set\ {\isacharparenleft}history\ i{\isacharparenright}\ {\isasymLongrightarrow}\ Broadcast\ {\isacharparenleft}Insert\ {\isacharparenleft}id{\isadigit{2}}{\isacharcomma}\ v{\isadigit{2}}{\isacharcomma}\ b{\isadigit{2}}{\isacharparenright}\ n{\isadigit{2}}{\isacharparenright}\ {\isasymin}\ set\ {\isacharparenleft}history\ j{\isacharparenright}\ {\isasymLongrightarrow}\ v{\isadigit{1}}\ {\isacharequal}\ v{\isadigit{2}}\ {\isasymand}\ n{\isadigit{1}}\ {\isacharequal}\ n{\isadigit{2}}{\isachardoublequoteclose}\isanewline
\ \ \isakeyword{assumes}\ allowed{\isacharunderscore}delete{\isacharcolon}\ {\isachardoublequoteopen}Broadcast\ {\isacharparenleft}Delete\ x{\isacharparenright}\ {\isasymin}\ set\ {\isacharparenleft}history\ i{\isacharparenright}\ {\isasymLongrightarrow}\ {\isacharparenleft}{\isasymexists}n{\isacharprime}\ v\ b{\isachardot}\ Deliver\ {\isacharparenleft}Insert\ {\isacharparenleft}x{\isacharcomma}\ v{\isacharcomma}\ b{\isacharparenright}\ n{\isacharprime}{\isacharparenright}\ {\isasymsqsubset}\isactrlsup i\ Broadcast\ {\isacharparenleft}Delete\ x{\isacharparenright}{\isacharparenright}{\isachardoublequoteclose}\isanewline
\end{isabellebody}


\begin{isabellebody}
\isanewline
\isacommand{lemma}\isamarkupfalse%
\ {\isacharparenleft}\isakeyword{in}\ rga{\isacharparenright}\ apply{\isacharunderscore}opers{\isacharunderscore}idx{\isacharunderscore}elems{\isacharcolon}\isanewline
\ \ \isakeyword{assumes}\ {\isachardoublequoteopen}es\ prefix\ of\ i{\isachardoublequoteclose}\isanewline
\ \ \ \ \ \ \isakeyword{and}\ {\isachardoublequoteopen}apply{\isacharunderscore}operations\ es\ {\isacharequal}\ Some\ xs{\isachardoublequoteclose}\isanewline
\ \ \ \ \isakeyword{shows}\ {\isachardoublequoteopen}fst\ {\isacharbackquote}\ set\ xs\ {\isacharequal}\ set\ {\isacharparenleft}indices\ es{\isacharparenright}{\isachardoublequoteclose}\isanewline
\end{isabellebody}


\begin{isabellebody}
\isanewline
\isacommand{theorem}\isamarkupfalse%
\ {\isacharparenleft}\isakeyword{in}\ rga{\isacharparenright}\ apply{\isacharunderscore}operations{\isacharunderscore}never{\isacharunderscore}fails{\isacharcolon}\isanewline
\ \ \isakeyword{assumes}\ {\isachardoublequoteopen}xs\ prefix\ of\ i{\isachardoublequoteclose}\isanewline
\ \ \isakeyword{shows}\ {\isachardoublequoteopen}apply{\isacharunderscore}operations\ xs\ {\isasymnoteq}\ None{\isachardoublequoteclose}\isanewline
\end{isabellebody}

\begin{isabellebody}
\isanewline
\isacommand{lemma}\isamarkupfalse%
\ {\isacharparenleft}\isakeyword{in}\ rga{\isacharparenright}\ node{\isacharunderscore}deliver{\isacharunderscore}messages{\isacharunderscore}distinct{\isacharcolon}\isanewline
\ \ \isakeyword{assumes}\ {\isachardoublequoteopen}xs\ prefix\ of\ i{\isachardoublequoteclose}\isanewline
\ \ \isakeyword{shows}\ {\isachardoublequoteopen}distinct\ {\isacharparenleft}node{\isacharunderscore}deliver{\isacharunderscore}messages\ xs{\isacharparenright}{\isachardoublequoteclose}\isanewline
\end{isabellebody}

\begin{isabellebody}
\isanewline
\isacommand{lemma}\isamarkupfalse%
\ {\isacharparenleft}\isakeyword{in}\ rga{\isacharparenright}\ hb{\isacharunderscore}consistent{\isacharunderscore}prefix{\isacharcolon}\isanewline
\ \ \isakeyword{assumes}\ {\isachardoublequoteopen}xs\ prefix\ of\ i{\isachardoublequoteclose}\isanewline
\ \ \ \ \isakeyword{shows}\ {\isachardoublequoteopen}hb{\isachardot}hb{\isacharunderscore}consistent\ {\isacharparenleft}node{\isacharunderscore}deliver{\isacharunderscore}messages\ xs{\isacharparenright}{\isachardoublequoteclose}\isanewline
\end{isabellebody}

\begin{isabellebody}
\isanewline
\isacommand{lemma}\isamarkupfalse%
\ {\isacharparenleft}\isakeyword{in}\ rga{\isacharparenright}\ concurrent{\isacharunderscore}operations{\isacharunderscore}commute{\isacharcolon}\isanewline
\ \ \isakeyword{assumes}\ {\isachardoublequoteopen}xs\ prefix\ of\ i{\isachardoublequoteclose}\isanewline
\ \ \isakeyword{shows}\ {\isachardoublequoteopen}hb{\isachardot}concurrent{\isacharunderscore}ops{\isacharunderscore}commute\ {\isacharparenleft}node{\isacharunderscore}deliver{\isacharunderscore}messages\ xs{\isacharparenright}{\isachardoublequoteclose}\isanewline
\end{isabellebody}

\begin{isabellebody}
\isanewline
\isacommand{corollary}\isamarkupfalse%
\ {\isacharparenleft}\isakeyword{in}\ rga{\isacharparenright}\ rga{\isacharunderscore}convergence{\isacharcolon}\isanewline
\ \ \isakeyword{assumes}\ {\isachardoublequoteopen}set\ {\isacharparenleft}node{\isacharunderscore}deliver{\isacharunderscore}messages\ xs{\isacharparenright}\ {\isacharequal}\ set\ {\isacharparenleft}node{\isacharunderscore}deliver{\isacharunderscore}messages\ ys{\isacharparenright}{\isachardoublequoteclose}\isanewline
\ \ \ \ \ \ \isakeyword{and}\ {\isachardoublequoteopen}xs\ prefix\ of\ i{\isachardoublequoteclose}\isanewline
\ \ \ \ \ \ \isakeyword{and}\ {\isachardoublequoteopen}ys\ prefix\ of\ j{\isachardoublequoteclose}\isanewline
\ \ \ \ \isakeyword{shows}\ {\isachardoublequoteopen}apply{\isacharunderscore}operations\ xs\ {\isacharequal}\ apply{\isacharunderscore}operations\ ys{\isachardoublequoteclose}\isanewline
\end{isabellebody}

%%%%%%%%%%%%%%%%%%%%%%%%%%%%%%%%%%%%%%%%%%%%%%%%%%%%%%%%%%%%%%%%%%%%%%%%%%%%%%%%
% Example
%%%%%%%%%%%%%%%%%%%%%%%%%%%%%%%%%%%%%%%%%%%%%%%%%%%%%%%%%%%%%%%%%%%%%%%%%%%%%%%%

\section{Example}
\label{sect.example}

%%%%%%%%%%%%%%%%%%%%%%%%%%%%%%%%%%%%%%%%%%%%%%%%%%%%%%%%%%%%%%%%%%%%%%%%%%%%%%%%
% Discussion
%%%%%%%%%%%%%%%%%%%%%%%%%%%%%%%%%%%%%%%%%%%%%%%%%%%%%%%%%%%%%%%%%%%%%%%%%%%%%%%%

\section{Discussion}
\label{sect.discussion}

%%%%%%%%%%%%%%%%%%%%%%%%%%%%%%%%%%%%%%%%%%%%%%%%%%%%%%%%%%%%%%%%%%%%%%%%%%%%%%%%
% Limitations
%%%%%%%%%%%%%%%%%%%%%%%%%%%%%%%%%%%%%%%%%%%%%%%%%%%%%%%%%%%%%%%%%%%%%%%%%%%%%%%%

\section{Limitations}
\label{sect.limitations}

%%%%%%%%%%%%%%%%%%%%%%%%%%%%%%%%%%%%%%%%%%%%%%%%%%%%%%%%%%%%%%%%%%%%%%%%%%%%%%%%
% Related Work
%%%%%%%%%%%%%%%%%%%%%%%%%%%%%%%%%%%%%%%%%%%%%%%%%%%%%%%%%%%%%%%%%%%%%%%%%%%%%%%%

\section{Related Work}
\label{sect.relatedwork}


\subsection{Formal Verification}

Zeller et al. have used Isabelle to formally specify and verify a number of state-based CRDTs~\cite{Zeller:2014fl}.

Some operational transformation functions have also been formally specified and verified using the
SPIKE theorem prover~\cite{Imine:2003ks,Imine:2006kn}, Coq~\cite{Sinchuk:2016cf}, and
Isabelle~\cite{Jungnickel:2015ua}. These efforts focus on proving that the transformation functions
satisfy given properties (such as the \emph{transformation properties} $\mathit{TP}_1$ and
$\mathit{TP}_2$~\cite{Oster:2006tr,Ressel:1996wx}).

control algorithm (also known as integration algorithm)

Prove commutativity properties of the transformation function, but do not formally relate these
properties to a particular network model. Informal reasoning is used to demonstrate that these
properties do indeed ensure convergence in all possible
executions~\cite{Suleiman:1998eu,Sun:1998vf}.


% \cite{Burckhardt:2014ft}
% \cite{Roh:2009ws}
% \cite{Bieniusa:2012gt}
% \cite{Wang:2015vo}
% \cite{Spiewak:2010vw}
% \cite{Google:2015vk}
% \cite{Lemonik:2016wh}
% \cite{Lamport:1978jq}
% \cite{Preguica:2012fx}
% \cite{Schwarz:1994gl}
% \cite{ParkerJr:1983jb}
% vector clocks~\cite{Fidge:1988tv,Raynal:1996jl}
% \cite{Defago:2004ji}
% \cite{Chandra:1996cp}
% \cite{Davidson:1985hv}
% \cite{Terry:1995dn}
% \cite{DeCandia:2007ui}

%%%%%%%%%%%%%%%%%%%%%%%%%%%%%%%%%%%%%%%%%%%%%%%%%%%%%%%%%%%%%%%%%%%%%%%%%%%%%%%%
% RGA_Network
%%%%%%%%%%%%%%%%%%%%%%%%%%%%%%%%%%%%%%%%%%%%%%%%%%%%%%%%%%%%%%%%%%%%%%%%%%%%%%%%

\section{Conclusion}
\label{sect.conclusion}

\subsection*{Acknowledgements}

\bibliography{references}{}

\end{document}
