\documentclass[acmlarge,review,anonymous]{acmart}\settopmatter{printfolios=true}

\bibliographystyle{ACM-Reference-Format}
\citestyle{acmauthoryear}
\usepackage[english]{babel}

\setcopyright{none} % For review submission

\begin{document}
\title{Formal Verification of Peer-to-Peer Collaborative Editing}
%\author{Victor~B.~F.~Gomes, Martin Kleppmann, Dominic P.~Mulligan,\\Alastair R. Beresford}
%\date{Computer Laboratory, University of Cambridge}


\maketitle

\begin{abstract}
To be completed...
\end{abstract}

%%%%%%%%%%%%%%%%%%%%%%%%%%%%%%%%%%%%%%%%%%%%%%%%%%%%%%%%%%%%%%%%%%%%%%%%%%%%%%%%
% Section
%%%%%%%%%%%%%%%%%%%%%%%%%%%%%%%%%%%%%%%%%%%%%%%%%%%%%%%%%%%%%%%%%%%%%%%%%%%%%%%%

\section{Introduction}
\label{sect.introduction}

Collaborative editing applications such as Google Docs~\cite{DayRichter:2010tt}, Microsoft Word
Online, Etherpad~\cite{Etherpad:2011um}, and Novell Vibe~\cite{Spiewak:2010vw} are increasing in
popularity. A common feature of these tools is that they allow several users to concurrently modify
a document without having to send the document back and forth manually (e.g., by email), and without
requiring any exclusive locking or manual resolution of merge conflicts.

However, all currently deployed collaborative editing systems rely on a central server to determine
a sequential order of edit operations, a design originally pioneered by the Jupiter
system~\cite{Nichols:1995fd}. This architecture has the advantage of simplifying the collaborative
editing algorithm by restricting the concurrency in the system.

However, the use of a central server introduces a number of limitations. It limits offline usage, preventing
a user from synchronising changes locally between, say, their smartphone and their laptop; it also
prevents a group of users from collaborating on a local network disconnected from the server. 
Since a central server model requires the server to sequentially order messages,
message meta data must be modifiable by the server, placing a high level of trust in the server
and limiting any confidentiality and integrity guarantees such a system can offer to users. %Is this too vague? 
Finally, a central server requires high availability and is also readily amenable to blocking and censorship. 
For sensitive scenarios, such as communication between dissidents of a repressive regime, such centralization is problematic.

Decentralized peer-to-peer systems therefore look attractive. In a decentralized setting, 
especially when the participating nodes are mobile devices, totally ordered broadcast of 
editing operations is prohibitively expensive~\cite{Attiya:2015dm}. Thus, there
has been significant interest in collaborative editing algorithms that work correctly in the face of
the increased concurrency encountered in peer-to-peer systems~\cite{Randolph:2015gj}.

There are two families of algorithms for collaborative editing: \emph{operational transformation}
(OT)~\cite{Ellis:1989ue,Ressel:1996wx,Oster:2006tr,Sun:1998vf,Sun:1998un,Suleiman:1998eu,Nichols:1995fd}
and \emph{conflict-free replicated datatypes}
(CRDTs)~\cite{Shapiro:2011wy,Roh:2011dw,Preguica:2009fz,Oster:2006wj,Weiss:2010hx,Nedelec:2013ky,Kleppmann:2016ve}.
Both allow a document to be modified concurrently on different replicas, with changes applied
immediately to the local copy, while asynchronously propagating changes to other replicas. The
goal of these algorithms is to ensure that for all concurrent executions, the replicas converge
toward the same state without any edits being lost, a property known as \emph{strong eventual
consistency}~\cite{Shapiro:2011un}.

However, these algorithms have a checkered history. OT algorithms have a reputation of being
difficult to understand and to implement correctly~\cite{Spiewak:2010vw}. Despite the fact that OT
has been studied for almost three decades~\cite{Ellis:1989ue}, few algorithms work correctly in a
peer-to-peer setting, and several published algorithms were later shown to violate their supposed
convergence guarantees~\cite{Imine:2003ks,Imine:2006kn}. It has even been proved that in the classic
formulation of OT it is impossible to achieve an important property required for convergence
in a peer-to-peer setting% (called $\mathit{TP}_2$ in the literature)
, and that additional parameters must be added to transformation functions
to make convergence possible~\cite{Randolph:2015gj}.

CRDTs are a more recent development~\cite{Shapiro:2011un}. While OT is based on transforming
non-commutative operations so that they have the same effect when reordered, CRDTs define operations
in a way that makes them commutative by design, making them more amenable to peer-to-peer settings
in which each node may apply edits in a different order. CRDTs also have attractive performance
characteristics~\cite{Mehdi:2011ke}.

To date there has been little formal verification of the correctness of CRDTs, and the
history of broken OT algorithms highlights the inadequacy of informal reasoning in this domain. In
this work we contribute to the formal basis of collaborative editing algorithms by using the
interactive proof assistant Isabelle to develop machine-checked proofs of the
correctness for CRDTs.

In particular, we study the Replicated Growable Array (RGA) CRDT~\cite{Roh:2011dw}, which represents
a collaboratively editable document as a sequence of characters. There are previous pen-and-paper
correctness proofs of RGA in the literature~\cite{Attiya:2016kh,Kleppmann:2016ve,Roh:2009ws}, but to
our knowledge, ours is the first mechanized proof of the convergence of RGA. The algorithm is
subtle~-- Attiya et al.\ recently wrote, ``the reason why RGA actually works has been a bit of a
mystery''~\cite{Attiya:2016kh}~-- which makes formal verification particularly important.

Our proof is structured in four modules:
\begin{enumerate}
    \item A general convergence theorem that applies in any system where concurrent operations are
        commutative;
    \item A formal model of a network protocol providing reliable, causally-ordered broadcast;
    \item An implementation of the RGA algorithm, and a proof that well-formed, concurrent insertion
        and deletion operations commute;
    \item A proof that when the RGA algorithm is executed in our network model, all possible
        executions are well-formed, and thus converge.
\end{enumerate}

In doing so, we go much further than previous correctness proofs of collaborative editing
algorithms. Previous formalizations of OT using theorem
provers~\cite{Imine:2003ks,Imine:2006kn,Sinchuk:2016cf,Jungnickel:2015ua} focus on proving that the
transformation functions satisfy given properties (such as the transformation properties
$\mathit{TP}_1$ and $\mathit{TP}_2$~\cite{Oster:2006tr,Ressel:1996wx}), and do not explicitly model
the network. A previous formalization of CRDTs~\cite{Zeller:2014fl}, also using Isabelle, considers
other datatypes (sets, registers, counters) but not the ordered sequence datatype provided by RGA.

By including a model of the network in our proof, we rule out a larger set of potential errors in
the algorithm that may result from the interaction of operation properties with assumptions about
the network. Moreover, our network model and convergence theorems are independent of any particular
CRDT, so they can be reused for correctness proofs of any other replicated datatype that is based on
operation commutativity, encompassing a wide range of CRDTs~\cite{Baquero:2014ed}.

Besides presenting the first machine-checked proof of the RGA algorithm, our main contribution in
this paper is to establish a modular toolkit of proof techniques and building blocks for
machine-checked correctness proofs of operation-based CRDTs. Our proofs are broken down into modules
with well-defined properties, allowing modules to be reused for proofs of new datatypes in future.
By making formal verification easier, we hope to provide a strong foundation for the development of
the next generation of algorithms for collaborative editing.

%TODO Various decentralised algorithms have been proposed and all but one (Oster 2006) have subsequently been shown to be incorrect.

%TODO Why do we want to demonstrate that CRDTs are somehow more worthy than OT of our attention?

%TODO Point out that we have a network model, when most proofs do not feature this aspect.


\section{Related Work}
\label{sect.relatedwork}


\subsection{Formal Verification}

Zeller et al. have used Isabelle to formally specify and verify a number of state-based CRDTs~\cite{Zeller:2014fl}.

Some operational transformation functions have also been formally specified and verified using the
SPIKE theorem prover~\cite{Imine:2003ks,Imine:2006kn}, Coq~\cite{Sinchuk:2016cf}, and
Isabelle~\cite{Jungnickel:2015ua}. These efforts focus on proving that the transformation functions
satisfy given properties (such as the \emph{transformation properties} $\mathit{TP}_1$ and
$\mathit{TP}_2$~\cite{Oster:2006tr,Ressel:1996wx}).

control algorithm (also known as integration algorithm)

Prove commutativity properties of the transformation function, but do not formally relate these
properties to a particular network model. Informal reasoning is used to demonstrate that these
properties do indeed ensure convergence in all possible
executions~\cite{Suleiman:1998eu,Sun:1998vf}.


% \cite{Burckhardt:2014ft}
% \cite{Roh:2009ws}
% \cite{Bieniusa:2012gt}
% \cite{Wang:2015vo}
% \cite{Spiewak:2010vw}
% \cite{Google:2015vk}
% \cite{Lemonik:2016wh}
% \cite{Lamport:1978jq}
% \cite{Preguica:2012fx}
% \cite{Schwarz:1994gl}
% \cite{ParkerJr:1983jb}
% vector clocks~\cite{Fidge:1988tv,Raynal:1996jl}
% \cite{Defago:2004ji}
% \cite{Chandra:1996cp}
% \cite{Davidson:1985hv}
% \cite{Terry:1995dn}
% \cite{DeCandia:2007ui}

\section{Conclusion}
\label{sect.conclusion}

\subsection*{Acknowledgements}

\bibliography{references}{}

\end{document}
