\section{Replicated Growable Array}
\label{sect.rga}

The RGA, introduced by \citet{Roh:2011dw}, is a replicated ordered list (sequence) datatype that supports \emph{insert} and \emph{delete} operations.
It can be used for collaborative editing of text by representing a string as an ordered list of characters.

The convergence of RGA has been proved by hand in previous work (see Section~\ref{sect.related.verification}); we now present the first (to our knowledge) mechanised proof that RGA satisfies the specification of SEC from Section~\ref{sect.abstract.convergence}.
We perform this proof within the causal broadcast model defined in Section~\ref{sect.network}, and without making any assumptions beyond the six aforementioned network axioms.
Since the axioms of our network model are easily justified, we have confidence in the correctness of our formalisation.
Our proof makes extensive use of the general-purpose framework that we have established in the last two sections.

\subsection{Specifying insertion and deletion}\label{sect.rga.spec}

In an ordered list, each insertion and deletion operation must identify the position at which the modification should take place.
In a non-replicated setting, the position is commonly expressed as an index into the list.
However, the index of a list element may change if other elements are concurrently inserted or deleted earlier in the list; this is the problem at the heart of Operational Transformation (see Section~\ref{sect.related.ot.crdts}).
Instead of using indexes, the RGA algorithm assigns a unique, immutable identifier to each list element.

Insertion operations place the new element \emph{after} an existing list element with a given ID, or at the head of the list if no ID is given.
Deletion operations refer to the ID of the list element that is to be deleted.
However, it is not safe for a deletion operation to completely remove a list element, because then a concurrent insertion after the deleted element would not be able to locate the insertion position.
Instead, the list retains \emph{tombstones}: a deletion operation merely sets a flag on a list element to mark it as deleted, but the element actually remains in the list.
A garbage collection process can be used to purge tombstones \cite{Roh:2011dw}, but we do not consider it here.

The RGA state at each node is a list of elements.
Each element is a triple consisting of the unique ID of the list element (of some type $\isacharprime\isa{id}$), the value inserted by the application (of some type $\isacharprime\isa{v}$), and a flag that indicates whether the element has been marked as deleted (of type $\isa{bool}$):
\begin{isabelle}
\isacommand{type{\isacharunderscore}synonym} {\isacharparenleft}{\isacharprime}id{\isacharcomma}\ {\isacharprime}v{\isacharparenright}\ elt\ {\isacharequal}\ {\isachardoublequoteopen}{\isacharprime}id\ {\isasymtimes}\ {\isacharprime}v\ {\isasymtimes}\ bool{\isachardoublequoteclose}%
\end{isabelle}

The $\isa{insert}$ function takes three parameters: the previous state of the list, the new element to insert, and optionally the ID of an existing element after which the new element should be inserted.
It returns the list with the new element inserted at the appropriate position, or $\isa{None}$ on failure, which occurs if there was no existing element with the given ID.
The function iterates over the list, and for each list element $\isa{x}$, it compares the ID (the first component of the $\isacharprime\isa{id} \mathbin{\isasymtimes} \isacharprime\isa{v} \mathbin{\isasymtimes} \isa{bool}$ triple, written $\isa{fst x}$) to the requested insertion position:
\begin{isabelle}
~~~~{\isachardoublequoteopen}insert\ {\isacharparenleft}x{\isacharhash}xs{\isacharparenright}\ \=e\ {\isacharparenleft}Some\ i{\isacharparenright}\ \={\isacharequal}\ {\isacharparenleft}\=\kill
\isacommand{fun}\ insert\ {\isacharcolon}{\isacharcolon}\ {\isachardoublequoteopen}{\isacharparenleft}{\isacharprime}id{\isacharcolon}{\isacharcolon}{\isacharbraceleft}linorder{\isacharbraceright}{\isacharcomma}\ {\isacharprime}v{\isacharparenright}\ elt\ list\ {\isasymRightarrow}\ {\isacharparenleft}{\isacharprime}id{\isacharcomma}\ {\isacharprime}v{\isacharparenright}\ elt\ {\isasymRightarrow}\ {\isacharprime}id\ option\ {\isasymRightarrow}\ {\isacharparenleft}{\isacharprime}id{\isacharcomma}\ {\isacharprime}v{\isacharparenright}\ elt\ list\ option{\isachardoublequoteclose}\\
\isakeyword{where}\\
~~~~{\isachardoublequoteopen}insert\ xs\ \>e\ None\ \>{\isacharequal}\ Some\ {\isacharparenleft}insert{\isacharunderscore}body\ xs\ e{\isacharparenright}{\isachardoublequoteclose}\ {\isacharbar}\\
~~~~{\isachardoublequoteopen}insert\ {\isacharbrackleft}{\isacharbrackright}\ \>e\ {\isacharparenleft}Some\ i{\isacharparenright}\ \>{\isacharequal}\ None{\isachardoublequoteclose}\ {\isacharbar}\\
~~~~{\isachardoublequoteopen}insert\ {\isacharparenleft}x{\isacharhash}xs{\isacharparenright}\ \>e\ {\isacharparenleft}Some\ i{\isacharparenright}\ \>{\isacharequal}\ {\isacharparenleft}\>if\ fst\ x\ {\isacharequal}\ i\ then\ Some\ {\isacharparenleft}x{\isacharhash}insert{\isacharunderscore}body\ xs\ e{\isacharparenright}\\
\>\>\>else\ insert\ xs\ e\ {\isacharparenleft}Some\ i{\isacharparenright}\ {\isasymbind}\ {\isacharparenleft}{\isasymlambda}t{\isachardot}\ Some\ {\isacharparenleft}x{\isacharhash}t{\isacharparenright}{\isacharparenright}{\isacharparenright}{\isachardoublequoteclose}
\end{isabelle}
When the insertion position is found (or, in the case of insertion at the head of the list, immediately), the function $\isa{insert-body}$ is invoked to perform the actual insertion:
\begin{isabelle}
~~~~{\isachardoublequoteopen}insert{\isacharunderscore}body\ {\isacharparenleft}x{\isacharhash}xs{\isacharparenright}\ \=\kill
\isacommand{fun} insert{\isacharunderscore}body\ {\isacharcolon}{\isacharcolon}\ {\isachardoublequoteopen}{\isacharparenleft}{\isacharprime}id{\isacharcolon}{\isacharcolon}{\isacharbraceleft}linorder{\isacharbraceright}{\isacharcomma}\ {\isacharprime}v{\isacharparenright}\ elt\ list\ {\isasymRightarrow}\ {\isacharparenleft}{\isacharprime}id{\isacharcomma}\ {\isacharprime}v{\isacharparenright}\ elt\ {\isasymRightarrow}\ {\isacharparenleft}{\isacharprime}id{\isacharcomma}\ {\isacharprime}v{\isacharparenright}\ elt\ list{\isachardoublequoteclose}\ \isakeyword{where}\\
~~~~{\isachardoublequoteopen}insert{\isacharunderscore}body\ {\isacharbrackleft}{\isacharbrackright} \>e\ {\isacharequal}\ {\isacharbrackleft}e{\isacharbrackright}{\isachardoublequoteclose}\ {\isacharbar}\\
~~~~{\isachardoublequoteopen}insert{\isacharunderscore}body\ {\isacharparenleft}x{\isacharhash}xs{\isacharparenright}\>e\ {\isacharequal}\ {\isacharparenleft}if\ fst\ x\ {\isacharless}\ fst\ e\ then\ e{\isacharhash}x{\isacharhash}xs\ else\ x{\isacharhash}insert{\isacharunderscore}body\ xs\ e{\isacharparenright}{\isachardoublequoteclose}
\end{isabelle}

In a non-replicated datatype it would be sufficient to insert the new element directly at the position found by the $\isa{insert}$ function.
However, a replicated setting is more difficult, because several nodes may concurrently insert new elements at the same position, and those insertion operations may be processed in a different order by different nodes.
In order to ensure that all nodes converge towards the same state (that is, the same order of list elements), we sort any concurrent insertions at the same position in descending order of the inserted elements' IDs.
This sorting is implemented in $\isa{insert-body}$ by skipping over any elements with an ID that is greater than that of the newly inserted element (the $\isa{fst x} > \isa{fst e}$ case), and then placing the new element before the first existing element with a lesser ID (the $\isa{fst x} < \isa{fst e}$ case).

Note that the type of IDs is specified as $\isacharprime\isa{id}\isacharcolon\isacharcolon\isacharbraceleft\isa{linorder}\isacharbraceright$, which means that we require the type $\isacharprime\isa{id}$ to have an associated total (linear) order.
$\isa{linorder}$ is the name of a type class supplied by the Isabelle/HOL library.
This annotation is required in order to be able to perform the comparison $\isa{fst x} < \isa{fst e}$ on IDs.
To be precise, RGA requires the total order of IDs to be consistent with causality, which can easily be achieved using the logical timestamps defined by \citet{Lamport:1978jq}.

The delete operation searches for the element with a given ID, and sets its flag to $\isa{True}$ to mark it as deleted:
\begin{isabelle}
~~~~{\isachardoublequoteopen}delete\ {\isacharparenleft}{\isacharparenleft}i{\isacharprime}{\isacharcomma}\ v{\isacharcomma}\ flag{\isacharparenright}{\isacharhash}xs{\isacharparenright}\ \=i\ {\isacharequal}\ {\isacharparenleft}\=\kill
\isacommand{fun}\ delete\ {\isacharcolon}{\isacharcolon}\ {\isachardoublequoteopen}{\isacharparenleft}{\isacharprime}id{\isacharcolon}{\isacharcolon}{\isacharbraceleft}linorder{\isacharbraceright}{\isacharcomma}\ {\isacharprime}v{\isacharparenright}\ elt\ list\ {\isasymRightarrow}\ {\isacharprime}id\ {\isasymRightarrow}\ {\isacharparenleft}{\isacharprime}id{\isacharcomma}\ {\isacharprime}v{\isacharparenright}\ elt\ list\ option{\isachardoublequoteclose}\ \isakeyword{where}\\
~~~~{\isachardoublequoteopen}delete\ {\isacharbrackleft}{\isacharbrackright}\>i\ {\isacharequal}\ None{\isachardoublequoteclose}\ {\isacharbar}\\
~~~~{\isachardoublequoteopen}delete\ {\isacharparenleft}{\isacharparenleft}i{\isacharprime}{\isacharcomma}\ v{\isacharcomma}\ flag{\isacharparenright}{\isacharhash}xs{\isacharparenright} \>i\ {\isacharequal}\ {\isacharparenleft}\>if\ i{\isacharprime}\ {\isacharequal}\ i\ then\ Some\ {\isacharparenleft}{\isacharparenleft}i{\isacharprime}{\isacharcomma}\ v{\isacharcomma}\ True{\isacharparenright}{\isacharhash}xs{\isacharparenright}\\
\>\>else\ delete\ xs\ i\ {\isasymbind}\ {\isacharparenleft}{\isasymlambda}t{\isachardot}\ Some\ {\isacharparenleft}{\isacharparenleft}i{\isacharprime}{\isacharcomma}v{\isacharcomma}flag{\isacharparenright}{\isacharhash}t{\isacharparenright}{\isacharparenright}{\isacharparenright}{\isachardoublequoteclose}%
\end{isabelle}
Note that the operations presented here are deliberately inefficient in order to make them easier to reason about.
One can see our implementations of $\isa{insert-body}$, $\isa{insert}$, and $\isa{delete}$ as functional specifications for RGAs, which could be optimised into more efficient algorithms using data refinement, if desired.

\subsection{Commutativity of insertion and deletion}

Recall from Section~\ref{sect.ops.commute} that in order to prove the convergence theorem we need to show that for the datatype in question, all its concurrent operations commute.
It is straightforward to demonstrate that $\isa{delete}$ always commutes with itself, on concurrent and non-concurrent operations alike:
\begin{isabelle}
\isacommand{lemma}\ delete{\isacharunderscore}commutes{\isacharcolon}\\
~~~~{\isachardoublequoteopen}delete\ xs\ i{\isadigit{1}}\ {\isasymbind}\ {\isacharparenleft}{\isasymlambda}ys{\isachardot}\ delete\ ys\ i{\isadigit{2}}{\isacharparenright}\ {\isacharequal}\ delete\ xs\ i{\isadigit{2}}\ {\isasymbind}\ {\isacharparenleft}{\isasymlambda}ys{\isachardot}\ delete\ ys\ i{\isadigit{1}}{\isacharparenright}{\isachardoublequoteclose}
\end{isabelle}

It is a little more complex to demonstrate that two $\isa{insert}$ operations commute.
Let $\isa{e1}$ and $\isa{e2}$ be the two new list elements being inserted, each of which is a $\isacharprime\isa{id} \mathbin{\isasymtimes} \isacharprime\isa{v} \mathbin{\isasymtimes} \isa{bool}$ triple.
Further, let $\isa{i1} \mathbin{\isacharcolon\isacharcolon} \isacharprime\isa{id option}$ be the position after which $\isa{e1}$ should be inserted (either $\isa{None}$ for the head of the list, or $\isa{Some i}$ where $\isa{i}$ is the ID of an existing list element), and similarly let $\isa{i2}$ be the position after which $\isa{e2}$ should be inserted.
Then the two insertions commute only under certain assumptions:
\begin{isabelle}
~~~~\isakeyword{assumes}\ \=\kill
\isacommand{lemma}\ insert{\isacharunderscore}commutes{\isacharcolon}\\
~~~~\isakeyword{assumes}\>{\isachardoublequoteopen}fst\ e{\isadigit{1}}\ {\isasymnoteq}\ fst\ e{\isadigit{2}}{\isachardoublequoteclose}\\
~~~~~~~~\isakeyword{and}\>{\isachardoublequoteopen}i{\isadigit{1}}\ {\isacharequal}\ None\ {\isasymor}\ i{\isadigit{1}}\ {\isasymnoteq}\ Some\ {\isacharparenleft}fst\ e{\isadigit{2}}{\isacharparenright}{\isachardoublequoteclose}\\
~~~~~~~~\isakeyword{and}\>{\isachardoublequoteopen}i{\isadigit{2}}\ {\isacharequal}\ None\ {\isasymor}\ i{\isadigit{2}}\ {\isasymnoteq}\ Some\ {\isacharparenleft}fst\ e{\isadigit{1}}{\isacharparenright}{\isachardoublequoteclose}\\
~~~~\isakeyword{shows}\>{\isachardoublequoteopen}insert\ xs\ e{\isadigit{1}}\ i{\isadigit{1}}\ {\isasymbind}\ {\isacharparenleft}{\isasymlambda}ys{\isachardot}\ insert\ ys\ e{\isadigit{2}}\ i{\isadigit{2}}{\isacharparenright}\ {\isacharequal}\ insert\ xs\ e{\isadigit{2}}\ i{\isadigit{2}}\ {\isasymbind}\ {\isacharparenleft}{\isasymlambda}ys{\isachardot}\ insert\ ys\ e{\isadigit{1}}\ i{\isadigit{1}}{\isacharparenright}{\isachardoublequoteclose}
\end{isabelle}
\noindent
That is, $\isa{i1}$ cannot refer to the ID of $\isa{e2}$ and vice versa, and the IDs of the two insertions must be distinct.
We prove later that these assumptions are indeed satisfied for all concurrent operations.
Finally, $\isa{delete}$ commutes with $\isa{insert}$ whenever the element to be deleted is not the same as the element to be inserted:
\begin{isabelle}
~~~~\isakeyword{assumes}\ \=\kill
\isacommand{lemma}\ insert{\isacharunderscore}delete{\isacharunderscore}commute{\isacharcolon}\\
~~~~\isakeyword{assumes}\>{\isachardoublequoteopen}i{\isadigit{2}}\ {\isasymnoteq}\ fst\ e{\isachardoublequoteclose}\\
~~~~\isakeyword{shows}\>{\isachardoublequoteopen}insert\ xs\ e\ i{\isadigit{1}}\ {\isasymbind}\ {\isacharparenleft}{\isasymlambda}ys{\isachardot}\ delete\ ys\ i{\isadigit{2}}{\isacharparenright}\ {\isacharequal}\ delete\ xs\ i{\isadigit{2}}\ {\isasymbind}\ {\isacharparenleft}{\isasymlambda}ys{\isachardot}\ insert\ ys\ e\ i{\isadigit{1}}{\isacharparenright}{\isachardoublequoteclose}
\end{isabelle}

\subsection{Embedding RGA in the network model}

In order to obtain a proof of the strong eventual consistency of RGA, we embed the insertion and deletion operations in the network model of Section~\ref{sect.network}.
We first define a datatype for operations (which are sent across the network in messages), and an interpretation function as introduced in Section~\ref{sect.ops.interpretation}:
\begin{isabelle}
~~~~{\isachardoublequoteopen}interpret{\isacharunderscore}opers\ {\isacharparenleft}Insert\ e\ n{\isacharparenright}\ \=\kill
\isacommand{datatype} {\isacharparenleft}{\isacharprime}id{\isacharcomma}\ {\isacharprime}v{\isacharparenright}\ operation\ {\isacharequal} Insert\ {\isachardoublequoteopen}{\isacharparenleft}{\isacharprime}id{\isacharcomma}\ {\isacharprime}v{\isacharparenright}\ elt{\isachardoublequoteclose}\ {\isachardoublequoteopen}{\isacharprime}id\ option{\isachardoublequoteclose}\ {\isacharbar} Delete\ {\isachardoublequoteopen}{\isacharprime}id{\isachardoublequoteclose}\\[4pt]
\isacommand{fun} interpret{\isacharunderscore}opers\ {\isacharcolon}{\isacharcolon}\ {\isachardoublequoteopen}{\isacharparenleft}{\isacharprime}id{\isacharcolon}{\isacharcolon}linorder{\isacharcomma}\ {\isacharprime}v{\isacharparenright}\ operation\ {\isasymRightarrow}\ {\isacharparenleft}{\isacharprime}id{\isacharcomma}\ {\isacharprime}v{\isacharparenright}\ elt\ list\ {\isasymRightarrow}\ {\isacharparenleft}{\isacharprime}id{\isacharcomma}\ {\isacharprime}v{\isacharparenright}\ elt\ list\ option{\isachardoublequoteclose}\\
\isakeyword{where}\\
~~~~{\isachardoublequoteopen}interpret{\isacharunderscore}opers\ {\isacharparenleft}Insert\ e\ n{\isacharparenright}\>xs\ \ {\isacharequal}\ insert\ xs\ e\ n{\isachardoublequoteclose}\ {\isacharbar}\\
~~~~{\isachardoublequoteopen}interpret{\isacharunderscore}opers\ {\isacharparenleft}Delete\ n{\isacharparenright}\>xs\ \ {\isacharequal}\ delete\ xs\ n{\isachardoublequoteclose}
\end{isabelle}

As discussed above, the validity of operations depends on some assumptions: IDs of insertion operations must be unique, and whenever an insertion or deletion operation refers to an existing list element, that element must exist.
As introduced in Section~\ref{sect.network.ops}, we can describe these requirements by using a predicate to specify what messages a node is allowed to broadcast when in a particular state:
\begin{isabelle}
~~~~~~~~\={\isacharparenleft}i{\isacharcomma}\ Insert\ e\ {\isacharparenleft}\=Some\ \=pos{\isacharparenright}\={\isacharparenright}\ \={\isasymRightarrow}\ fst\ e\ {\isacharequal}\ i\ {\isasymand}\ \=\kill
\isacommand{definition}\ valid{\isacharunderscore}rga{\isacharunderscore}msg\ {\isacharcolon}{\isacharcolon}\ {\isachardoublequoteopen}{\isacharparenleft}{\isacharprime}id{\isacharcomma}\ {\isacharprime}v{\isacharparenright}\ elt\ list\ {\isasymRightarrow}\ {\isacharprime}id\ {\isasymtimes}\ {\isacharparenleft}{\isacharprime}id{\isacharcolon}{\isacharcolon}linorder{\isacharcomma}\ {\isacharprime}v{\isacharparenright}\ operation\ {\isasymRightarrow}\ bool{\isachardoublequoteclose}\ \isakeyword{where}\\
~~~~{\isachardoublequoteopen}valid{\isacharunderscore}rga{\isacharunderscore}msg\ list\ msg\ {\isasymequiv}\ case\ msg\ of\\
\>{\isacharparenleft}i{\isacharcomma}\ Insert\ e \>None \>\>{\isacharparenright}\ \>{\isasymRightarrow}\ fst\ e\ {\isacharequal}\ i\ {\isacharbar}\\
\>{\isacharparenleft}i{\isacharcomma}\ Insert\ e\ {\isacharparenleft}Some\ pos{\isacharparenright}{\isacharparenright}\ {\isasymRightarrow}\ fst\ e\ {\isacharequal}\ i\ {\isasymand}\ pos\ {\isasymin}\ set\ {\isacharparenleft}map\ fst\ list{\isacharparenright}\ {\isacharbar}\\
\>{\isacharparenleft}i{\isacharcomma}\ Delete \>\>pos \>{\isacharparenright} \>{\isasymRightarrow} \>pos\ {\isasymin}\ set\ {\isacharparenleft}map\ fst\ list{\isacharparenright}{\isachardoublequoteclose}
\end{isabelle}
We can now define RGA by extending $\isa{network-with-constrained-ops}$. The interpretation function is instantiated with $\isa{interpret-opers}$, the initial state with the empty list $\isacharbrackleft\,\isacharbrackright$, and the validity predicate with $\isa{valid-rga-msg}$:
\begin{isabelle}
\isacommand{locale}\ rga\ {\isacharequal}\ network{\isacharunderscore}with{\isacharunderscore}constrained{\isacharunderscore}ops\ {\isacharunderscore}\ interpret{\isacharunderscore}opers\ {\isachardoublequoteopen}{\isacharbrackleft}{\isacharbrackright}{\isachardoublequoteclose}\ valid{\isacharunderscore}rga{\isacharunderscore}msg
\end{isabelle}

Within this locale, we prove that whenever an insertion or deletion operation $\isa{op2}$ references an existing list element, there is always a prior insertion operation $\isa{op1}$ that created the element being referenced:
\begin{isabelle}
~~~~\isakeyword{assumes}\ \={\isachardoublequoteopen}n\ {\isacharequal}\ None\ {\isasymor}\ {\isacharparenleft}{\isasymexists}e{\isacharprime}\ n{\isacharprime}{\isachardot}\ \=\kill
\isacommand{lemma}\ allowed{\isacharunderscore}insert{\isacharcolon}\\
~~~~\isakeyword{assumes}\>{\isachardoublequoteopen}Broadcast\ {\isacharparenleft}Insert\ e\ n{\isacharparenright}\ {\isasymin}\ set\ {\isacharparenleft}history\ i{\isacharparenright}{\isachardoublequoteclose}\\
~~~~\isakeyword{shows}\>{\isachardoublequoteopen}n\ {\isacharequal}\ None\ {\isasymor}\ {\isacharparenleft}{\isasymexists}e{\isacharprime}\ n{\isacharprime}{\isachardot}\>n\ {\isacharequal}\ Some\ {\isacharparenleft}fst\ e{\isacharprime}{\isacharparenright}\ {\isasymand}\\
\>\>Deliver\ {\isacharparenleft}Insert\ e{\isacharprime}\ n{\isacharprime}{\isacharparenright}\ {\isasymsqsubset}\isactrlsup i\ Broadcast\ {\isacharparenleft}Insert\ e\ n{\isacharparenright}{\isacharparenright}{\isachardoublequoteclose}\\[4pt]
\isacommand{lemma}\ allowed{\isacharunderscore}delete{\isacharcolon}\\
~~~~\isakeyword{assumes}\>{\isachardoublequoteopen}Broadcast\ {\isacharparenleft}Delete\ x{\isacharparenright}\ {\isasymin}\ set\ {\isacharparenleft}history\ i{\isacharparenright}{\isachardoublequoteclose}\\
~~~~\isakeyword{shows}\>{\isachardoublequoteopen}{\isasymexists}n{\isacharprime}\ v\ b{\isachardot}\ Deliver\ {\isacharparenleft}Insert\ {\isacharparenleft}x{\isacharcomma}\ v{\isacharcomma}\ b{\isacharparenright}\ n{\isacharprime}{\isacharparenright}\ {\isasymsqsubset}\isactrlsup i\ Broadcast\ {\isacharparenleft}Delete\ x{\isacharparenright}{\isachardoublequoteclose}
\end{isabelle}
Since the network ensures causally ordered delivery, all nodes must deliver the insertion $\isa{op1}$ before the dependent operation $\isa{op2}$.
Hence we show that in all cases where operations do not commute, one operation happens before another.
Conversely, whenever operations are concurrent, we show that they commute:
\begin{isabelle}
\isacommand{theorem}\ concurrent{\isacharunderscore}operations{\isacharunderscore}commute{\isacharcolon}\\
~~~~\isakeyword{shows}\ {\isachardoublequoteopen}hb{\isachardot}concurrent{\isacharunderscore}ops{\isacharunderscore}commute\ {\isacharparenleft}node{\isacharunderscore}deliver{\isacharunderscore}messages\ {\isacharparenleft}history\ i{\isacharparenright}{\isacharparenright}{\isachardoublequoteclose}
\end{isabelle}
Furthermore, although the type signature of the interpretation function allows an operation to fail by returning $\isa{None}$, we can prove that this failure case is never reached in any execution of the network:
\begin{isabelle}
\isacommand{theorem}\ apply{\isacharunderscore}operations{\isacharunderscore}never{\isacharunderscore}fails{\isacharcolon}\\
~~~~\isakeyword{shows}\ {\isachardoublequoteopen}hb.apply{\isacharunderscore}operations\ {\isacharparenleft}node{\isacharunderscore}deliver{\isacharunderscore}messages\ {\isacharparenleft}history\ i{\isacharparenright}{\isacharparenright}\ {\isasymnoteq}\ None{\isachardoublequoteclose}
\end{isabelle}
It is now easy to show that the $\isa{rga}$ locale satisfies all of the requirements of the abstract $\isa{strong-eventual-consistency}$ specification (Section~\ref{sect.abstract.sec.spec}), which demonstrates formally that RGA provides SEC.
