\section{High-level proof strategy}
\label{sect.high-level.proof.strategy}

Since our formalisation of distributed algorithms goes into greater depth than prior work on strong eventual consistency, it is important to have a structure that keeps the proofs manageable.
Our approach breaks the proof into simple modules with cleanly defined properties---called \emph{locales}, a standard sectioning mechanism of Isabelle/HOL that will be described in Section~\ref{subsect.an.overview.of.isabelle} below---and composes them in order to describe more complex objects.
This locale structure is illustrated in Figure~\ref{fig.proof.structure} and explained below.
%See Section~\ref{sect.isabelle.locales} for more detail on locales.

\begin{figure}
\centering
\begin{tikzpicture}[auto,scale=1.0]

\tikzstyle{locale}=[draw,anchor=base,minimum width=30pt,text height=8pt,text depth=3pt]

\node (preorder) at (0,7.5) [locale] {$\isa{preorder}$};
\node (hb)       at (0,6)   [locale] {$\isa{happens-before}$};
\node (sec)      at (0,4)   [locale] {$\isa{strong-eventual-consistency}$};
\node (hist)     at (5,6)   [locale] {$\isa{node-histories}$};
\node (network)  at (5,5)   [locale] {$\isa{network}$};
\node (causal)   at (5,4)   [locale] {$\isa{causal-network}$};
\node (netops)   at (5,3)   [locale] {$\isa{network-with-ops}$};
\node (netops2)  at (7,1.8) [locale] {$\isa{network-with-constrained-ops}$};
\node (counter)  at (1,0)   [locale] {$\isa{counter}$};
\node (orset)    at (3,0)   [locale] {$\isa{orset}$};
\node (rga)      at (5,0)   [locale] {$\isa{rga}$};

\begin{scope}[thick,dotted]
    \path [draw] (-2.5,6.8) -- ( 10,6.8);
    \path [draw] (-2.5,0.8) -- ( 10,0.8);
    \path [draw] ( 2.8,0.8) -- (2.8,6.8);
\end{scope}

\begin{scope}[left,text width=5cm,text ragged left,font=\footnotesize]
    \node at (10, 7.2) {Isabelle standard library};
    \node at (10, 6.2) {Network model\\(Section \ref{sect.network})};
\end{scope}
\begin{scope}[text width=5cm,font=\footnotesize]
    \node at (0, 1.4) {Abstract convergence\\(Section \ref{sect.abstract.convergence})};
    \node at (0, 0.2) {Example CRDTs\\(Sections \ref{sect.rga} and \ref{sect.simple.crdts})};
\end{scope}

\begin{scope}[>=open triangle 60]
    \tikzstyle{every path}=[thick,->]
    \draw (hb) to (preorder);
    \draw (sec) to (hb);
    \draw (network) to (hist);
    \draw (causal) to (network);
    \draw (netops) to (causal);
    \draw (node cs:name=netops2,angle=164) to [out=90,in=270] (node cs:name=netops, angle=340);
    \draw (node cs:name=counter,angle=50)  to [out=90,in=270] (node cs:name=netops, angle=200);
    \draw (node cs:name=orset,  angle=50)  to [out=90,in=270] (node cs:name=netops2,angle=191);
    \draw (node cs:name=rga,    angle=50)  to [out=90,in=270] (node cs:name=netops2,angle=210);
    \tikzstyle{every path}=[thick,dashed,->]
    \draw (node cs:name=causal, angle=160) to [out=90,in=270] (node cs:name=hb, angle=340);
    \draw (node cs:name=counter,angle=90)  to [out=90,in=270] (node cs:name=sec,angle=210);
    \draw (node cs:name=orset,  angle=90)  to [out=90,in=270] (node cs:name=sec,angle=270);
    \draw (node cs:name=rga,    angle=90)  to [out=90,in=270] (node cs:name=sec,angle=330);
\end{scope}
\end{tikzpicture}

\caption{The main locales (modules) of our proof, and the relationships between them.
Solid arrows indicate a more specialised locale that extends a more general locale (like extending interfaces in OOP).
Dashed arrows indicate a sublocale that satisfies the assumptions of the superlocale (like implementing an interface in OOP).
}\label{fig.proof.structure}
\end{figure}

By lines of code, more than half of our proof is used to construct a general-purpose model of consistency in distributed systems, described in Section~\ref{sect.abstract.convergence}, and an axiomatic model of a computer network, described in Section~\ref{sect.network}, with both modules independent of any particular replication algorithm.
The remainder describes a formalisation of three CRDTs and their proofs of correctness, described in Sections~\ref{sect.rga} and~\ref{sect.simple.crdts}.
By keeping the general-purpose modules abstract and implementation-independent, we construct a reusable library of specifications and theorems.

We describe our formalisation of strong eventual consistency in Section~\ref{sect.abstract.convergence}.
In particular, we define what we mean by convergence, and prove an \emph{abstract convergence theorem}, which shows that the state of nodes converges if concurrent operations commute.
We are able to prove this fact without mentioning networks or any particular CRDT, but merely by reasoning about the ordering and properties of operations.
This definition constitutes a formal specification of what we mean by strong eventual consistency.

In Section~\ref{sect.network} we describe an axiomatic model of asynchronous networks.
The definition of the network is important because it allows us to prove that the desired properties hold in \emph{all} possible network behaviours, and that we are not making any dangerous assumptions that might be violated---an aspect that has dogged previous verification efforts for related algorithms (see Section~\ref{sect.related.verification}).
The network is the only part of our proof in which we make any axiomatic assumptions, and we show in Section~\ref{sect.network} that our assumptions are realistic, reflecting both standard conventions for modelling distributed systems, and the practical realities of network protocols today.
We then prove that our network satisfies the ordering properties required by the abstract convergence theorem of Section~\ref{sect.abstract.convergence}, and thus deduce a convergence theorem for our network model.

We use the general-purpose theorems and definitions from Sections~\ref{sect.abstract.convergence} and~\ref{sect.network} to prove the strong eventual consistency properties of concrete algorithms.
In Section~\ref{sect.rga} we describe our formalisation of the Replicated Growable Array (RGA), a CRDT for ordered lists.
We first show how to implement the RGA's insert and delete operations, with proofs that each operation commutes with itself, and that all operations commute with each other.
Insertion and deletion only commute under various conditions, so we prove that these conditions are satisfied in all possible network behaviours, and thus we obtain a concrete convergence theorem for our RGA implementation.
Next, in Section~\ref{sect.simple.crdts}, we demonstrate the generality of our proof framework with definitions of two simple CRDTs: a Counter and an Observed-Remove Set.
%We show that with the previously established tooling in place, it is very easy to prove that these algorithms implement the abstract specification of strong eventual consistency.

As illustrated in Figure~\ref{fig.proof.structure}, the $\isa{counter}$, $\isa{orset}$, and $\isa{rga}$ locales can use the definitions and lemmas of the network model because they extend that model.
We then prove that all three locales satisfy the abstract specification $\isa{strong-eventual-consistency}$, and therefore show that these algorithms provide strong eventual consistency.
