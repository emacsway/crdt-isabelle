\section{Two other CRDTs: Counter and Set}
\label{sect.simple.crdts}

We close the technical body of our paper with a demonstration that our network and abstract convergence theorem truly form a `framework' for mechanically verifying the convergence of operation-based CRDTs.
In particular, we demonstrate the convergence of two well-known CRDTs---the increment-decrement counter, and the observed-removed set---again as corollaries of our abstract convergence theorem.
This bolsters our claim that the network and abstract convergence layers of our mechanisation are reusable components, and fully independent of any one CRDT implementation.

\subsection{Increment-decrement counter}
\label{subsect.increment-decrement.counter}

The increment-decrement counter is perhaps the simplest CRDT, and a paradigmatic example of a replicated data structure with commutative operations.
As the name suggests, the data structure supports two operations: $\isa{increment}$ and $\isa{decrement}$ which respectively increment and decrement a shared integer counter:
\vspace{0.375em}
\begin{isabellebody}
\ \ \ \ \ \ \ \ \isacommand{datatype}\ operation {\isacharequal}\ Increment\ {\isacharbar}\ Decrement
\end{isabellebody}
\vspace{0.375em}
Providing a semantics for these two operations is straightforward, so we only present the function implementing the increment operation, here, eliding the corresponding implementation of decrement:
\vspace{0.375em}
\begin{isabellebody}
\ \ \ \ \ \ \ \ \isacommand{definition}\ increment\ {\isacharcolon}{\isacharcolon}\ {\isachardoublequoteopen}int\ {\isasymRightarrow}\ int{\isachardoublequoteclose}\ \isakeyword{where}\isanewline
\ \ \ \ \ \ \ \ \ \ {\isachardoublequoteopen}increment\ x\ {\isasymequiv}\ {\isadigit{1}}\ {\isacharplus}\ x{\isachardoublequoteclose}
\end{isabellebody}
\vspace{0.375em}
Note that neither $\isa{increment}$ nor $\isa{decrement}$ fail on under- or overflow, as both functions are defined on a type of unbounded (mathematical) integers.
We could also have implemented the counter using fixed bitwidth integers---e.g. signed 32- or 64-bit machine words---with wrap-around on overflow, or some other similar design choice.
Any reasonable alternative design would not have impacted the proofs presented here.
Interpreting the $\isa{operation}$ data type and lifting it into a (partial) state transformer is a straightforward exercise in pattern matching:
\vspace{0.375em}
\begin{isabellebody}
\ \ \ \ \ \ \ \ \isacommand{fun}\ interpret{\isacharunderscore}operation\ {\isacharcolon}{\isacharcolon}\ {\isachardoublequoteopen}operation\ {\isasymRightarrow}\ int\ {\isasymrightharpoonup}\ int{\isachardoublequoteclose}\ \isakeyword{where}\isanewline
\ \ \ \ \ \ \ \ \ \ {\isachardoublequoteopen}interpret{\isacharunderscore}operation\ Increment\ {\isacharequal}\ Some\ o\ increment{\isachardoublequoteclose}\ {\isacharbar}\isanewline
\ \ \ \ \ \ \ \ \ \ {\isachardoublequoteopen}interpret{\isacharunderscore}operation\ Decrement\ {\isacharequal}\ Some\ o\ decrement{\isachardoublequoteclose}
\end{isabellebody}
\vspace{0.375em}
Here, $\isa{f}\ \isa{o}\ \isa{g}$ denotes the composition of functions $\isa{f}$ and $\isa{g}$.
Note that as neither of the $\isa{increment}$ and $\isa{decrement}$ operations fail, each must be composed with the $\isa{Some}$ constructor of the $\isa{option}$ type to effect the lifting into a partial state transformer, suitable for use with our $\isa{network}{\isacharunderscore}\isa{with}{\isacharunderscore}\isa{ops}$ locale.

We next consider commutation of increment and decrement.
Showing that both operations commute with each other, and with themselves, is an easy exercise in applying Isabelle's proof automation.
We present only a single commutation lemma, here, eliding the rest:
\vspace{0.375em}
\begin{isabellebody}
\ \ \ \ \ \ \ \ \isacommand{lemma}\ increment{\isacharunderscore}decrement{\isacharunderscore}commute{\isacharcolon}\isanewline
\ \ \ \ \ \ \ \ \ \ \isakeyword{shows}\ {\isachardoublequoteopen}increment\ {\isacharparenleft}decrement\ x{\isacharparenright}\ {\isacharequal}\ decrement\ {\isacharparenleft}increment\ x{\isacharparenright}{\isachardoublequoteclose}
\end{isabellebody}
\vspace{0.375em}
Unlike more complex CRDTs, such as the Replicated Growable Array discussed in Section~\ref{sect.rga}, the operations of the increment-decrement counter commute unconditionally.
As a result, the increment-decrement counter will converge in any execution of a causal asynchronous network, without restriction on possible network behaviours.
We capture this by creating a CRDT-specific locale, $\isa{counter}$, which is just a thin veneer over the $\isa{network}{\isacharunderscore}\isa{with}{\isacharunderscore}\isa{ops}$ locale, specialising the types and providing a fixed initial state (here, $\isa{0}$):
\vspace{0.375em}
\begin{isabellebody}
\ \ \ \ \ \ \ \ \isacommand{locale}\ counter\ {\isacharequal}\ network{\isacharunderscore}with{\isacharunderscore}ops\ {\isacharunderscore}\ {\isacharunderscore}\ interpret{\isacharunderscore}operation\ {\isadigit{0}}
\end{isabellebody}
\vspace{0.375em}
Within this locale, we can then show that all concurrent delivered messages of a prefix of a node's local history, within a $\isa{counter}$ network commute.
Compare to the Replicated Growable Array case, this is straightforward, as operations commute unconditionally:
\vspace{0.375em}
\begin{isabellebody}
\ \ \ \ \ \ \ \ \isacommand{lemma}\ {\isacharparenleft}\isakeyword{in}\ counter{\isacharparenright}\ concurrent{\isacharunderscore}operations{\isacharunderscore}commute{\isacharcolon}\isanewline
\ \ \ \ \ \ \ \ \ \ \isakeyword{assumes}\ {\isachardoublequoteopen}xs\ prefix\ of\ i{\isachardoublequoteclose}\isanewline
\ \ \ \ \ \ \ \ \ \ \isakeyword{shows}\ {\isachardoublequoteopen}hb{\isachardot}concurrent{\isacharunderscore}ops{\isacharunderscore}commute\ {\isacharparenleft}node{\isacharunderscore}deliver{\isacharunderscore}messages\ xs{\isacharparenright}{\isachardoublequoteclose}
\end{isabellebody}
\vspace{0.375em}
From this, and our abstract convergence theorems, we can show convergence for the increment-decrement counter, which is capture by $\isa{counter}{\isacharunderscore}\isa{convergence}$:
\vspace{0.375em}
\begin{isabellebody}
\ \ \ \ \ \ \ \ \isacommand{corollary}\ {\isacharparenleft}\isakeyword{in}\ counter{\isacharparenright}\ counter{\isacharunderscore}convergence{\isacharcolon}\isanewline
\ \ \ \ \ \ \ \ \ \ \isakeyword{assumes}\ {\isachardoublequoteopen}set\ {\isacharparenleft}node{\isacharunderscore}deliver{\isacharunderscore}messages\ xs{\isacharparenright}\ {\isacharequal}\ set\ {\isacharparenleft}node{\isacharunderscore}deliver{\isacharunderscore}messages\ ys{\isacharparenright}{\isachardoublequoteclose}\isanewline
\ \ \ \ \ \ \ \ \ \ \ \ \ \ \isakeyword{and}\ {\isachardoublequoteopen}xs\ prefix\ of\ i{\isachardoublequoteclose}\ \isakeyword{and}\ {\isachardoublequoteopen}ys\ prefix\ of\ j{\isachardoublequoteclose}\isanewline
\ \ \ \ \ \ \ \ \ \ \ \ \isakeyword{shows}\ {\isachardoublequoteopen}apply{\isacharunderscore}operations\ xs\ {\isacharequal}\ apply{\isacharunderscore}operations\ ys{\isachardoublequoteclose}\isanewline
\end{isabellebody}

\subsection{Observed-removed set}
\label{subsect.observed-removed.set}

\begin{isabellebody}
\isacommand{fun}\isamarkupfalse%
\ opt{\isacharunderscore}default\ {\isacharcolon}{\isacharcolon}\ {\isachardoublequoteopen}{\isacharprime}a\ option\ {\isasymRightarrow}\ {\isacharprime}a\ {\isasymRightarrow}\ {\isacharprime}a{\isachardoublequoteclose}\ {\isacharparenleft}\isakeyword{infix}\ {\isachardoublequoteopen}default\ to{\isachardoublequoteclose}\ {\isadigit{5}}{\isadigit{0}}{\isacharparenright}\ \isakeyword{where}\isanewline
\ \ {\isachardoublequoteopen}opt{\isacharunderscore}default\ {\isacharparenleft}Some\ x{\isacharparenright}\ {\isacharunderscore}\ {\isacharequal}\ x{\isachardoublequoteclose}\ {\isacharbar}\isanewline
\ \ {\isachardoublequoteopen}opt{\isacharunderscore}default\ \ None\ \ \ \ x\ {\isacharequal}\ x{\isachardoublequoteclose}\isanewline
\end{isabellebody}

\begin{isabellebody}
\isacommand{datatype}\isamarkupfalse%
\ {\isacharprime}elem\ operation\isanewline
\ \ {\isacharequal}\ Add\ {\isacharprime}elem\isanewline
\ \ {\isacharbar}\ Rem\ {\isacharprime}elem\isanewline
\end{isabellebody}

\begin{isabellebody}
\isacommand{type{\isacharunderscore}synonym}\isamarkupfalse%
\ {\isacharprime}elem\ state\ {\isacharequal}\ {\isachardoublequoteopen}{\isacharprime}elem\ {\isasymrightharpoonup}\ {\isacharprime}elem\ operation\ set{\isachardoublequoteclose}\isanewline
\end{isabellebody}

\begin{isabellebody}
\isacommand{locale}\isamarkupfalse%
\ orset{\isacharunderscore}base\ {\isacharequal}\ network{\isacharunderscore}with{\isacharunderscore}ops\ msg{\isacharunderscore}id\ history\ interp\ {\isachardoublequoteopen}Map{\isachardot}empty{\isachardoublequoteclose}\isanewline
\ \ \isakeyword{for}\ msg{\isacharunderscore}id\ {\isacharcolon}{\isacharcolon}\ {\isachardoublequoteopen}{\isacharprime}elem\ operation\ {\isasymRightarrow}\ {\isacharprime}a{\isachardoublequoteclose}\isanewline
\ \ \isakeyword{and}\ history\ {\isacharcolon}{\isacharcolon}\ {\isachardoublequoteopen}nat\ {\isasymRightarrow}\ {\isacharprime}elem\ operation\ event\ list{\isachardoublequoteclose}\isanewline
\ \ \isakeyword{and}\ interp\ {\isacharcolon}{\isacharcolon}\ {\isachardoublequoteopen}{\isacharprime}elem\ operation\ {\isasymRightarrow}\ {\isacharprime}elem\ state\ {\isasymrightharpoonup}\ {\isacharprime}elem\ state{\isachardoublequoteclose}\isanewline
\end{isabellebody}

\begin{isabellebody}
\isacommand{definition}\isamarkupfalse%
\ op{\isacharunderscore}elem\ \isakeyword{where}\isanewline
\ \ {\isachardoublequoteopen}op{\isacharunderscore}elem\ oper\ {\isasymequiv}\ case\ oper\ of\ Add\ e\ {\isasymRightarrow}\ e\ {\isacharbar}\ Rem\ e\ {\isasymRightarrow}\ e{\isachardoublequoteclose}\isanewline
\end{isabellebody}

\begin{isabellebody}
\isacommand{definition}\isamarkupfalse%
\ {\isacharparenleft}\isakeyword{in}\ orset{\isacharunderscore}base{\isacharparenright}\ interpret{\isacharunderscore}op\ {\isacharcolon}{\isacharcolon}\ {\isachardoublequoteopen}{\isacharprime}elem\ operation\ {\isasymRightarrow}\ {\isacharprime}elem\ state\ {\isasymrightharpoonup}\ {\isacharprime}elem\ state{\isachardoublequoteclose}\ {\isacharparenleft}{\isachardoublequoteopen}{\isasymlangle}{\isacharunderscore}{\isasymrangle}{\isachardoublequoteclose}\ {\isacharbrackleft}{\isadigit{0}}{\isacharbrackright}\ {\isadigit{1}}{\isadigit{0}}{\isadigit{0}}{\isadigit{0}}{\isacharparenright}\ \isakeyword{where}\isanewline
\ \ {\isachardoublequoteopen}interpret{\isacharunderscore}op\ oper\ state\ {\isacharequal}\ {\isacharparenleft}\isanewline
\ \ \ \ let\ before\ {\isacharequal}\ {\isacharparenleft}state\ {\isacharparenleft}op{\isacharunderscore}elem\ oper{\isacharparenright}{\isacharparenright}\ default\ to\ {\isacharbraceleft}{\isacharbraceright}\ in\isanewline
\ \ \ \ let\ after\ {\isacharequal}\ case\ oper\ of\isanewline
\ \ \ \ \ \ Add\ e\ {\isasymRightarrow}\ before\ {\isasymunion}\ {\isacharbraceleft}oper{\isacharbraceright}\ {\isacharbar}\isanewline
\ \ \ \ \ \ Rem\ e\ {\isasymRightarrow}\ {\isacharbraceleft}add\ {\isasymin}\ before{\isachardot}\ {\isasymnot}\ hb\ add\ oper{\isacharbraceright}\isanewline
\ \ \ \ in\ Some\ {\isacharparenleft}state\ {\isacharparenleft}{\isacharparenleft}op{\isacharunderscore}elem\ oper{\isacharparenright}\ {\isasymmapsto}\ after{\isacharparenright}{\isacharparenright}{\isacharparenright}{\isachardoublequoteclose}\isanewline
\end{isabellebody}

\begin{isabellebody}
\isacommand{locale}\isamarkupfalse%
\ orset\ {\isacharequal}\ orset{\isacharunderscore}base\ {\isacharunderscore}\ {\isacharunderscore}\ {\isachardoublequoteopen}orset{\isacharunderscore}base{\isachardot}interpret{\isacharunderscore}op\ history{\isachardoublequoteclose}\isanewline
\end{isabellebody}

\begin{isabellebody}
\isacommand{lemma}\isamarkupfalse%
\ map{\isacharunderscore}assign{\isacharunderscore}commute{\isacharcolon}\isanewline
\ \ \isakeyword{assumes}\ {\isachardoublequoteopen}k{\isadigit{1}}\ {\isasymnoteq}\ k{\isadigit{2}}{\isachardoublequoteclose}\isanewline
\ \ \isakeyword{shows}\ {\isachardoublequoteopen}{\isacharparenleft}{\isasymlambda}x{\isachardot}\ Some\ {\isacharparenleft}x{\isacharparenleft}k{\isadigit{1}}\ {\isasymmapsto}\ f\ {\isacharparenleft}x\ k{\isadigit{1}}\ default\ to\ {\isacharbraceleft}{\isacharbraceright}{\isacharparenright}{\isacharcomma}\ k{\isadigit{2}}\ {\isasymmapsto}\ g\ {\isacharparenleft}x\ k{\isadigit{2}}\ default\ to\ {\isacharbraceleft}{\isacharbraceright}{\isacharparenright}{\isacharparenright}{\isacharparenright}{\isacharparenright}\ {\isacharequal}\isanewline
\ \ \ \ \ \ \ \ \ {\isacharparenleft}{\isasymlambda}x{\isachardot}\ Some\ {\isacharparenleft}x{\isacharparenleft}k{\isadigit{2}}\ {\isasymmapsto}\ g\ {\isacharparenleft}x\ k{\isadigit{2}}\ default\ to\ {\isacharbraceleft}{\isacharbraceright}{\isacharparenright}{\isacharcomma}\ k{\isadigit{1}}\ {\isasymmapsto}\ f\ {\isacharparenleft}x\ k{\isadigit{1}}\ default\ to\ {\isacharbraceleft}{\isacharbraceright}{\isacharparenright}{\isacharparenright}{\isacharparenright}{\isacharparenright}{\isachardoublequoteclose}\isanewline
\end{isabellebody}

\begin{isabellebody}
\isacommand{lemma}\isamarkupfalse%
\ {\isacharparenleft}\isakeyword{in}\ orset{\isacharparenright}\ add{\isacharunderscore}add{\isacharunderscore}commute{\isacharcolon}\isanewline
\ \ \isakeyword{shows}\ {\isachardoublequoteopen}{\isasymlangle}Add\ e{\isadigit{1}}{\isasymrangle}\ {\isasymrhd}\ {\isasymlangle}Add\ e{\isadigit{2}}{\isasymrangle}\ {\isacharequal}\ {\isasymlangle}Add\ e{\isadigit{2}}{\isasymrangle}\ {\isasymrhd}\ {\isasymlangle}Add\ e{\isadigit{1}}{\isasymrangle}{\isachardoublequoteclose}\isanewline
\end{isabellebody}

\begin{isabellebody}
\isacommand{lemma}\isamarkupfalse%
\ {\isacharparenleft}\isakeyword{in}\ orset{\isacharparenright}\ add{\isacharunderscore}rem{\isacharunderscore}commute{\isacharcolon}\isanewline
\ \ \isakeyword{assumes}\ {\isachardoublequoteopen}{\isasymnot}\ hb\ {\isacharparenleft}Add\ e{\isadigit{1}}{\isacharparenright}\ {\isacharparenleft}Rem\ e{\isadigit{2}}{\isacharparenright}\ {\isasymand}\ {\isasymnot}\ hb\ {\isacharparenleft}Rem\ e{\isadigit{2}}{\isacharparenright}\ {\isacharparenleft}Add\ e{\isadigit{1}}{\isacharparenright}{\isachardoublequoteclose}\isanewline
\ \ \isakeyword{shows}\ {\isachardoublequoteopen}{\isasymlangle}Add\ e{\isadigit{1}}{\isasymrangle}\ {\isasymrhd}\ {\isasymlangle}Rem\ e{\isadigit{2}}{\isasymrangle}\ {\isacharequal}\ {\isasymlangle}Rem\ e{\isadigit{2}}{\isasymrangle}\ {\isasymrhd}\ {\isasymlangle}Add\ e{\isadigit{1}}{\isasymrangle}{\isachardoublequoteclose}
\end{isabellebody}

\begin{isabellebody}
\isacommand{lemma}\isamarkupfalse%
\ {\isacharparenleft}\isakeyword{in}\ orset{\isacharparenright}\ rem{\isacharunderscore}rem{\isacharunderscore}commute{\isacharcolon}\isanewline
\ \ \isakeyword{shows}\ {\isachardoublequoteopen}{\isasymlangle}Rem\ e{\isadigit{1}}{\isasymrangle}\ {\isasymrhd}\ {\isasymlangle}Rem\ e{\isadigit{2}}{\isasymrangle}\ {\isacharequal}\ {\isasymlangle}Rem\ e{\isadigit{2}}{\isasymrangle}\ {\isasymrhd}\ {\isasymlangle}Rem\ e{\isadigit{1}}{\isasymrangle}{\isachardoublequoteclose}
\end{isabellebody}

\begin{isabellebody}
\isacommand{lemma}\isamarkupfalse%
\ {\isacharparenleft}\isakeyword{in}\ orset{\isacharparenright}\ concurrent{\isacharunderscore}operations{\isacharunderscore}commute{\isacharcolon}\isanewline
\ \ \isakeyword{assumes}\ {\isachardoublequoteopen}xs\ prefix\ of\ i{\isachardoublequoteclose}\isanewline
\ \ \isakeyword{shows}\ {\isachardoublequoteopen}hb{\isachardot}concurrent{\isacharunderscore}ops{\isacharunderscore}commute\ {\isacharparenleft}node{\isacharunderscore}deliver{\isacharunderscore}messages\ xs{\isacharparenright}{\isachardoublequoteclose}
\end{isabellebody}

\begin{isabellebody}
\isacommand{theorem}\isamarkupfalse%
\ {\isacharparenleft}\isakeyword{in}\ orset{\isacharparenright}\ convergence{\isacharcolon}\isanewline
\ \ \isakeyword{assumes}\ {\isachardoublequoteopen}set\ {\isacharparenleft}node{\isacharunderscore}deliver{\isacharunderscore}messages\ xs{\isacharparenright}\ {\isacharequal}\ set\ {\isacharparenleft}node{\isacharunderscore}deliver{\isacharunderscore}messages\ ys{\isacharparenright}{\isachardoublequoteclose}\isanewline
\ \ \ \ \ \ \isakeyword{and}\ {\isachardoublequoteopen}xs\ prefix\ of\ i{\isachardoublequoteclose}\isanewline
\ \ \ \ \ \ \isakeyword{and}\ {\isachardoublequoteopen}ys\ prefix\ of\ j{\isachardoublequoteclose}\isanewline
\ \ \ \ \isakeyword{shows}\ {\isachardoublequoteopen}apply{\isacharunderscore}operations\ xs\ {\isacharequal}\ apply{\isacharunderscore}operations\ ys{\isachardoublequoteclose}\isanewline
\end{isabellebody}