\section{An Introduction to Isabelle}
\label{subsect.an.overview.of.isabelle}

To aid the understanding of later sections, we provide a brief introduction to the key concepts and syntax of Isabelle.
Readers who are already familiar with Isabelle may skip to Section~\ref{sect.abstract.convergence}.
A more detailed introduction can be found in the standard tutorial material~\cite{DBLP:books/sp/NipkowK14}.

\subsubsection*{Syntax of expressions}\label{sect.isabelle.syntax}

Isabelle/HOL is a logic with a strict, polymorphic type system resembling that of mainstream functional programming languages.
\emph{Function types} are written $\tau_1 \Rightarrow \tau_2$, and are inhabited by \emph{total} functions, mapping elements of $\tau_1$ to elements of $\tau_2$.
We write $\tau_1 \times \tau_2$ for the \emph{product type} of $\tau_1$ and $\tau_2$, inhabited by pairs of elements of type $\tau_1$ and $\tau_2$, respectively.
In a similar fashion to Standard ML and OCaml, but differing from Haskell, \emph{type operators} are applied to arguments in reverse order, and therefore write $\tau\ \isa{list}$ and $\tau\ \isa{set}$ for the type of lists of elements of type $\tau$, and the type of mathematical (i.e., potentially infinite) sets of type $\tau$, respectively.
Type variables are written in lowercase, and preceded with a prime: ${\isacharprime}a \Rightarrow {\isacharprime}a$ denotes the type of a polymorphic identity function, for example.
\emph{Tagged union} types are introduced with the $\isacommand{datatype}$ keyword, with constructors of these types usually written with an initial upper case letter.

In Isabelle/HOL's term language we write $\isa{t} \mathbin{::} \tau$ for a \emph{type ascription}, constraining the type of the term $\isa{t}$ to the type $\tau$.
We write $\lambda{x}.\: t$ for an anonymous function mapping an argument $\isa{x}$ to $\isa{t(x)}$, and write the application of term $\isa{t}$ with function type to an argument $\isa{u}$ as $\isa{t\ u}$, as usual.
Terms of list type are introduced using one of two constructors: the empty list $[\,]$ or `nil', and the infix operator $\isa{\#}$ which is pronounced `cons', and which prepends an element to an existing list.
We use $[t_1, \ldots, t_n]$ as syntactic sugar for a list literal, and $\isa{xs} \mathbin{\isacharat} \isa{ys}$ to express the concatenation (appending) of two lists $\isa{xs}$ and $\isa{ys}$.
We write $\{\,\}$ for the empty set, and use usual mathematical notation for set union, disjunction, membership tests, and so on: $\isa{t} \cup \isa{u}$, $\isa{t} \cap \isa{u}$, and $\isa{x} \in \isa{t}$.

Terms with type $\isa{bool}$ are called \emph{formulae}, writing $\isa{True}$ and $\isa{False}$ for the logical truthity and falsity constants, respectively.
We write $\isa{t} \longrightarrow \isa{u}$, $\isa{t} \wedge \isa{u}$, and $\isa{t} \vee \isa{u}$ for material implication, conjunction, and disjunction, respectively, and write $\neg t$ for the negation of a formula.
Isabelle/HOL's equality constant is polymorphic: we write $t = u$ for an assertion of equality between two terms of the same type.
We write $\forall{x}.\: t$ and $\exists{x}.\: t$ for universal and existential quantification---and write $\forall{x{\in}t}.\: u$ and $\exists{x{\in}t}.\: u$ for their bounded forms, restricted to members of a set $\isa{t}$.
An alternative implication arrow $\isa{t} \Longrightarrow \isa{u}$ is used by Isabelle in certain contexts; it is subtly different from standard implication $\isa{t} \longrightarrow \isa{u}$, but for purposes of an intuitive understanding, the two forms of implication can be regarded as equivalent.

\subsubsection*{Definitions and theorems}\label{sect.isabelle.definitions}

New non-recursive definitions are entered into Isabelle's global context using the $\mathbf{definition}$ keyword.
Recursive functions are defined using the $\mathbf{fun}$ keyword, and support pattern matching on the parameters.
All functions are total, and therefore every recursive function must be proved to always terminate.
The termination proofs in this work are generated automatically by Isabelle itself.

Inductive relations are defined with the $\mathbf{inductive}$ keyword.
For example, the definition
\vspace{0.375em}
\begin{isabellebody}
\ \ \ \ \ \ \ \ \isacommand{inductive} only-fives\ {\isacharcolon}{\isacharcolon}\ {\isachardoublequoteopen}nat\ list\ {\isasymRightarrow}\ bool{\isachardoublequoteclose}\ \isakeyword{where}\isanewline
\ \ \ \ \ \ \ \ \ \ {\isachardoublequoteopen}only-fives\ {\isacharbrackleft}{\isacharbrackright}{\isachardoublequoteclose}\ {\isacharbar}\isanewline
\ \ \ \ \ \ \ \ \ \ {\isachardoublequoteopen}{\isasymlbrakk}\ only-fives\ xs\ {\isasymrbrakk}\ {\isasymLongrightarrow}\ only-fives {\isacharparenleft}5\#xs{\isacharparenright}{\isachardoublequoteclose}
\end{isabellebody}
\vspace{0.375em}
\noindent
introduces a new constant $\isa{only-fives}$ of type $\isa{nat list} \Rightarrow \isa{bool}$.
The two clauses in the body of the definition enumerate the conditions under which $\isa{only-fives}\ \isa{xs}$ is true, for arbitrary $\isa{xs}$: firstly, $\isa{only-fives}$ is true for the empty list; and secondly, if you know that $\isa{only-fives}\ \isa{xs}$ is true for some $\isa{xs}$, then you can deduce that $\isa{only-fives}\ (5\#\isa{xs})$ (i.e., $\isa{xs}$ prefixed with a new list element that is the number 5) is also true.
Moreover, $\isa{only-fives}\ \isa{xs}$ is true in no other circumstances---$\isa{only-fives}$ is the \emph{smallest} relation closed under the rules defining it.
In short, the clauses defining $\isa{only-fives}$ above state that $\isa{only-fives}\ \isa{xs}$ holds exactly in the case where $\isa{xs}$ is a (potentially empty) list containing only repeated copies of the natural number $5$.

Lemmas, theorems, and corollaries can be asserted using the $\isacommand{lemma}$, $\isacommand{theorem}$, and $\isacommand{corollary}$ keywords, respectively.
There is no semantic difference between these keywords in Isabelle.
For example,
\vspace{0.375em}
\begin{isabellebody}
\ \ \ \ \ \ \ \ \isacommand{theorem} only-fives-concat{\isacharcolon}\isanewline
\ \ \ \ \ \ \ \ \ \ \isakeyword{assumes}\ only-fives\ xs \isakeyword{and}\ only-fives\ ys \isanewline
\ \ \ \ \ \ \ \ \ \ \isakeyword{shows}\ only-fives (xs \isacharat ys)
\end{isabellebody}
\vspace{0.375em}
\noindent
conjectures that if $\isa{xs}$ and $\isa{ys}$ are both lists of fives, then their concatenation $xs \mathbin{\isacharat} ys$ is also a list of fives.
Isabelle then requires that this claim be proved by using one of its proof methods, for example by induction.
Some proofs can be automated, whilst others require the user to provide explicit reasoning steps.
The theorem is assigned a name, here $\isa{only-fives-concat}$, so that it may be referenced in later proofs.

\subsubsection*{Locales}\label{sect.isabelle.locales}

Lastly, we use \emph{locales}---or local theories~\cite{DBLP:conf/tphol/KammullerWP99,DBLP:conf/types/HaftmannW08}---extensively to structure the proof, as shown in Figure~\ref{fig.proof.structure}.
A declaration of the form
\vspace{0.375em}
\begin{isabellebody}
\ \ \ \ \ \ \ \ \isacommand{locale} foo = \isanewline
\ \ \ \ \ \ \ \ \ \ \isakeyword{fixes}\ f\ {\isacharcolon}{\isacharcolon}\ {\isachardoublequoteopen}{\isacharprime}a\ {\isasymRightarrow}\ {\isacharprime}a{\isachardoublequoteclose}\isanewline
\ \ \ \ \ \ \ \ \ \ \isakeyword{assumes} {\isachardoublequoteopen}f\ x = x{\isachardoublequoteclose}
\end{isabellebody}
\vspace{0.375em}
\noindent
introduces a locale, with a fixed, typed constant $\isa{f}$, and an associated law that states that $\isa{f}$ is the identity function.
Functions and constants may now be defined, and theorems conjectured and proved, within the context of the $\isa{foo}$ theory.
This is indicated syntactically by writing $(\isacommand{in}\ \isa{foo})$ before the name of the constant being defined, or the theorem being conjectured, at the point of definition or conjecture.
Any function, constant, or theorem, marked in this way may make reference to $\isa{f}$, or the fact that $\isa{f}\ \isa{x} = \isa{x}$ for all $\isa{x}$.
\emph{Interpreting} a locale---such as $\isa{foo}$ above---involves providing a concrete implementation of $\isa{f}$ coupled with a proof that the concrete implementation satisfies the associated law.
Once interpreted, all functions, definitions, and theorems made within the $\isa{foo}$ locale become available to use for that concrete implementation.
